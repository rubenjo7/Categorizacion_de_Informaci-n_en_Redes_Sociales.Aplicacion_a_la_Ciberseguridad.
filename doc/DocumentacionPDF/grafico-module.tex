%
% API Documentation for Categorizacion de Informacion en Redes Sociales. Aplicacion a la Ciberseguridad.
% Module grafico
%
% Generated by epydoc 3.0.1
% [Sat Jun  3 09:18:27 2017]
%

%%%%%%%%%%%%%%%%%%%%%%%%%%%%%%%%%%%%%%%%%%%%%%%%%%%%%%%%%%%%%%%%%%%%%%%%%%%
%%                          Module Description                           %%
%%%%%%%%%%%%%%%%%%%%%%%%%%%%%%%%%%%%%%%%%%%%%%%%%%%%%%%%%%%%%%%%%%%%%%%%%%%

    \index{grafico \textit{(module)}|(}
\section{Module grafico}

    \label{grafico}
Este módulo contiene la clase \textbf{Grafico}.

\textbf{Author:} Rubén Jiménez Ortega.



\textbf{Version:} 1.0



\textbf{Copyright:} Copyright (C) 2017 by Rubén Jiménez Ortega.




%%%%%%%%%%%%%%%%%%%%%%%%%%%%%%%%%%%%%%%%%%%%%%%%%%%%%%%%%%%%%%%%%%%%%%%%%%%
%%                               Variables                               %%
%%%%%%%%%%%%%%%%%%%%%%%%%%%%%%%%%%%%%%%%%%%%%%%%%%%%%%%%%%%%%%%%%%%%%%%%%%%

  \subsection{Variables}

    \vspace{-1cm}
\hspace{\varindent}\begin{longtable}{|p{\varnamewidth}|p{\vardescrwidth}|l}
\cline{1-2}
\cline{1-2} \centering \textbf{Name} & \centering \textbf{Description}& \\
\cline{1-2}
\endhead\cline{1-2}\multicolumn{3}{r}{\small\textit{continued on next page}}\\\endfoot\cline{1-2}
\endlastfoot\raggedright v\-e\-n\-t\-a\-n\-a\-M\-a\-n\-e\-j\-o\-G\-r\-a\-f\-i\-c\-o\- & \raggedright Esta varible almacenará la ventana que maneja el Gráfico, para 
          así poder cambiar el estado de los botones.

\textbf{Value:} 
{\tt ManejoGrafico()}&\\
\cline{1-2}
\raggedright t\-e\-r\-m\-i\-n\-a\-d\-o\- & \raggedright Esta varible es falsa normalmente, salvo que se termine el 
          estudio de la lista de seguidores de un usuario.

\textbf{Value:} 
{\tt False}&\\
\cline{1-2}
\raggedright i\-n\-i\-c\-i\-o\- & \raggedright Esta varible siempre vale True a no ser que se pause, pare o 
          termine el grafo.

\textbf{Value:} 
{\tt True}&\\
\cline{1-2}
\raggedright G\- & \raggedright Inicialización del Grafo.

\textbf{Value:} 
{\tt nx.Graph()}&\\
\cline{1-2}
\raggedright n\-u\-m\-e\-r\-o\-P\-a\-u\-s\-a\- & \raggedright Para saber por que posición vamos en caso de que se pause el 
          grafo.

\textbf{Value:} 
{\tt 0}&\\
\cline{1-2}
\raggedright n\-o\-d\-o\-R\-a\-i\-z\- & \raggedright Valor del nodo raíz. O bien se pasa al inicio del programa o bien
          cuando cambiamos de nodo.

\textbf{Value:} 
{\tt ""}&\\
\cline{1-2}
\raggedright s\-e\-g\-u\-n\-d\-o\-N\-i\-v\-e\-l\- & \raggedright Cuando estamos estudiando el segundo nivel, el usuario.

\textbf{Value:} 
{\tt ""}&\\
\cline{1-2}
\raggedright r\-o\-w\-\_\-c\-o\-u\-n\-t\- & \raggedright Número de columnas que tiene un usuario en sus seguidos, o lo que
          es lo mismo, el número de seguidos.

\textbf{Value:} 
{\tt 0}&\\
\cline{1-2}
\raggedright r\-o\-w\-\_\-c\-o\-u\-n\-t\-\_\-n\-o\-d\-o\-\_\-r\-a\-i\-z\- & \raggedright Número de columnas que tiene el usuario raíz en sus seguidos, o 
          lo que es lo mismo, el número de seguidos.

\textbf{Value:} 
{\tt 0}&\\
\cline{1-2}
\raggedright c\-o\-n\-t\-a\-d\-o\-r\- & \raggedright Para ver por donde vamos.

\textbf{Value:} 
{\tt 0}&\\
\cline{1-2}
\raggedright n\-u\-m\-e\-r\-o\-A\-l\-g\-u\-n\-a\-s\-C\-a\-t\-e\-g\-o\-r\-i\-a\-s\- & \raggedright Esta varible almacena el numero de usuarios que hay cuando no se 
          seleccionan todas la categorias.

\textbf{Value:} 
{\tt 0}&\\
\cline{1-2}
\raggedright l\-i\-s\-t\-a\-N\-o\-d\-o\-s\- & \raggedright Se van añadiendo los nodos cuando no se han seleccionado todas 
          las categorías.

\textbf{Value:} 
{\tt []}&\\
\cline{1-2}
\raggedright f\-i\-n\- & \raggedright Indica cuando hemos terminado el grafo.

\textbf{Value:} 
{\tt False}&\\
\cline{1-2}
\end{longtable}


%%%%%%%%%%%%%%%%%%%%%%%%%%%%%%%%%%%%%%%%%%%%%%%%%%%%%%%%%%%%%%%%%%%%%%%%%%%
%%                           Class Description                           %%
%%%%%%%%%%%%%%%%%%%%%%%%%%%%%%%%%%%%%%%%%%%%%%%%%%%%%%%%%%%%%%%%%%%%%%%%%%%

    \index{grafico \textit{(module)}!grafico.Grafico \textit{(class)}|(}
\subsection{Class Grafico}

    \label{grafico:Grafico}
Esta clase contiene todos los métodos relacionados con el Grafo.


%%%%%%%%%%%%%%%%%%%%%%%%%%%%%%%%%%%%%%%%%%%%%%%%%%%%%%%%%%%%%%%%%%%%%%%%%%%
%%                                Methods                                %%
%%%%%%%%%%%%%%%%%%%%%%%%%%%%%%%%%%%%%%%%%%%%%%%%%%%%%%%%%%%%%%%%%%%%%%%%%%%

  \subsubsection{Methods}

    \label{grafico:Grafico:setInicio}
    \index{grafico \textit{(module)}!grafico.Grafico \textit{(class)}!grafico.Grafico.setInicio \textit{(method)}}

    \vspace{0.5ex}

\hspace{.8\funcindent}\begin{boxedminipage}{\funcwidth}

    \raggedright \textbf{setInicio}(\textit{self}, \textit{inicioNuevo})

    \vspace{-1.5ex}

    \rule{\textwidth}{0.5\fboxrule}
\setlength{\parskip}{2ex}
    Modifica el valor de la variable inicio.

\setlength{\parskip}{1ex}
      \textbf{Parameters}
      \vspace{-1ex}

      \begin{quote}
        \begin{Ventry}{xxxxxxxxxxx}

          \item[inicioNuevo]

          Nueva valor que adquiere \textit{inicio}.

            {\it (type=boolean)}

        \end{Ventry}

      \end{quote}

    \end{boxedminipage}

    \label{grafico:Grafico:setNodoRaiz}
    \index{grafico \textit{(module)}!grafico.Grafico \textit{(class)}!grafico.Grafico.setNodoRaiz \textit{(method)}}

    \vspace{0.5ex}

\hspace{.8\funcindent}\begin{boxedminipage}{\funcwidth}

    \raggedright \textbf{setNodoRaiz}(\textit{self}, \textit{nuevoNodo})

    \vspace{-1.5ex}

    \rule{\textwidth}{0.5\fboxrule}
\setlength{\parskip}{2ex}
    Modifica el valor del nodo Raíz.

\setlength{\parskip}{1ex}
      \textbf{Parameters}
      \vspace{-1ex}

      \begin{quote}
        \begin{Ventry}{xxxxxxxxx}

          \item[nuevoNodo]

          Nueva valor que adquiere \textit{nodoRaiz}.

            {\it (type=str)}

        \end{Ventry}

      \end{quote}

    \end{boxedminipage}

    \label{grafico:Grafico:segundoNivel}
    \index{grafico \textit{(module)}!grafico.Grafico \textit{(class)}!grafico.Grafico.segundoNivel \textit{(method)}}

    \vspace{0.5ex}

\hspace{.8\funcindent}\begin{boxedminipage}{\funcwidth}

    \raggedright \textbf{segundoNivel}(\textit{self}, \textit{nuevoSegundoNivel})

    \vspace{-1.5ex}

    \rule{\textwidth}{0.5\fboxrule}
\setlength{\parskip}{2ex}
    Modifica el valor del segundoNivel.

\setlength{\parskip}{1ex}
      \textbf{Parameters}
      \vspace{-1ex}

      \begin{quote}
        \begin{Ventry}{xxxxxxxxxxxxxxxxx}

          \item[nuevoSegundoNivel]

          Nueva valor que adquiere \textit{segundoNivel}.

            {\it (type=str)}

        \end{Ventry}

      \end{quote}

    \end{boxedminipage}

    \label{grafico:Grafico:setNumeroAlgunasCategorias}
    \index{grafico \textit{(module)}!grafico.Grafico \textit{(class)}!grafico.Grafico.setNumeroAlgunasCategorias \textit{(method)}}

    \vspace{0.5ex}

\hspace{.8\funcindent}\begin{boxedminipage}{\funcwidth}

    \raggedright \textbf{setNumeroAlgunasCategorias}(\textit{self}, \textit{numero})

    \vspace{-1.5ex}

    \rule{\textwidth}{0.5\fboxrule}
\setlength{\parskip}{2ex}
    Modifica el valor de numeroAlgunasCategorias.

\setlength{\parskip}{1ex}
      \textbf{Parameters}
      \vspace{-1ex}

      \begin{quote}
        \begin{Ventry}{xxxxxx}

          \item[numero]

          Nueva valor que adquiere \textit{numeroAlgunasCategorias}.

            {\it (type=int)}

        \end{Ventry}

      \end{quote}

    \end{boxedminipage}

    \label{grafico:Grafico:setNumero}
    \index{grafico \textit{(module)}!grafico.Grafico \textit{(class)}!grafico.Grafico.setNumero \textit{(method)}}

    \vspace{0.5ex}

\hspace{.8\funcindent}\begin{boxedminipage}{\funcwidth}

    \raggedright \textbf{setNumero}(\textit{self}, \textit{numeroNuevo})

    \vspace{-1.5ex}

    \rule{\textwidth}{0.5\fboxrule}
\setlength{\parskip}{2ex}
    Modifica el valor de numero.

\setlength{\parskip}{1ex}
      \textbf{Parameters}
      \vspace{-1ex}

      \begin{quote}
        \begin{Ventry}{xxxxxxxxxxx}

          \item[numeroNuevo]

          Nueva valor que adquiere \textit{numero}.

            {\it (type=int)}

        \end{Ventry}

      \end{quote}

    \end{boxedminipage}

    \label{grafico:Grafico:setContador}
    \index{grafico \textit{(module)}!grafico.Grafico \textit{(class)}!grafico.Grafico.setContador \textit{(method)}}

    \vspace{0.5ex}

\hspace{.8\funcindent}\begin{boxedminipage}{\funcwidth}

    \raggedright \textbf{setContador}(\textit{self}, \textit{contadorNuevo})

    \vspace{-1.5ex}

    \rule{\textwidth}{0.5\fboxrule}
\setlength{\parskip}{2ex}
    Modifica el valor de contador.

\setlength{\parskip}{1ex}
      \textbf{Parameters}
      \vspace{-1ex}

      \begin{quote}
        \begin{Ventry}{xxxxxxxxxxxxx}

          \item[contadorNuevo]

          Nueva valor que adquiere \textit{contador}.

            {\it (type=int)}

        \end{Ventry}

      \end{quote}

    \end{boxedminipage}

    \label{grafico:Grafico:leerDocumentoTodasCategorias}
    \index{grafico \textit{(module)}!grafico.Grafico \textit{(class)}!grafico.Grafico.leerDocumentoTodasCategorias \textit{(method)}}

    \vspace{0.5ex}

\hspace{.8\funcindent}\begin{boxedminipage}{\funcwidth}

    \raggedright \textbf{leerDocumentoTodasCategorias}(\textit{self}, \textit{G}, \textit{numero}, \textit{row\_count}, \textit{usu})

    \vspace{-1.5ex}

    \rule{\textwidth}{0.5\fboxrule}
\setlength{\parskip}{2ex}
    Lee la lista de seguidos del usuario que se pasa por parámetro, tras 
    esto, en cada iteracción del bucle va añadiendo una relación nueva que 
    se añade al grafo. Tras acabar la iteracción se sale de la función para
    hacer que el grafo vaya creciendo progresivamente. Cuando termina la 
    lista de un usuario pone terminado a True y pasa al segundo nivel.

\setlength{\parskip}{1ex}
      \textbf{Parameters}
      \vspace{-1ex}

      \begin{quote}
        \begin{Ventry}{xxxxxxxxx}

          \item[G]

          Grafo al que añadiremos los nodos.

            {\it (type=nx.Graph())}

          \item[numero]

          Para saber por donde vamos.

            {\it (type=int)}

          \item[row\_count]

          Número de columnas en el archivo del usuario en cuestión.

            {\it (type=int)}

          \item[usu]

          Usuario del cual leemos sus seguidos.

            {\it (type=str)}

        \end{Ventry}

      \end{quote}

    \end{boxedminipage}

    \label{grafico:Grafico:sacarSeguidoresSegundoNivel}
    \index{grafico \textit{(module)}!grafico.Grafico \textit{(class)}!grafico.Grafico.sacarSeguidoresSegundoNivel \textit{(method)}}

    \vspace{0.5ex}

\hspace{.8\funcindent}\begin{boxedminipage}{\funcwidth}

    \raggedright \textbf{sacarSeguidoresSegundoNivel}(\textit{self}, \textit{usuario})

    \vspace{-1.5ex}

    \rule{\textwidth}{0.5\fboxrule}
\setlength{\parskip}{2ex}
    Leemmos los seguidos del usuario pasado por parámetro. A partir de 
    estos, se irán sacando las listas de los usuarios a segundo nivel. Una 
    vez terminado, rompemos la interacción.

\setlength{\parskip}{1ex}
      \textbf{Parameters}
      \vspace{-1ex}

      \begin{quote}
        \begin{Ventry}{xxxxxxx}

          \item[usuario]

          Usuario a partir del cual leemos sus seguidos, y de estos sacamos
          sus seguidos.

            {\it (type=str)}

        \end{Ventry}

      \end{quote}

    \end{boxedminipage}

    \label{grafico:Grafico:leerDocumentoAlgunasCategorias}
    \index{grafico \textit{(module)}!grafico.Grafico \textit{(class)}!grafico.Grafico.leerDocumentoAlgunasCategorias \textit{(method)}}

    \vspace{0.5ex}

\hspace{.8\funcindent}\begin{boxedminipage}{\funcwidth}

    \raggedright \textbf{leerDocumentoAlgunasCategorias}(\textit{self}, \textit{G}, \textit{row\_count}, \textit{usu}, \textit{checkAnimales}, \textit{checkRopa}, \textit{checkTerrorismo}, \textit{checkComida}, \textit{checkSc})

    \vspace{-1.5ex}

    \rule{\textwidth}{0.5\fboxrule}
\setlength{\parskip}{2ex}
    Lee la lista de seguidos del usuario que se pasa por parámetro, tras 
    esto, se chequean que botones están activos para contar como una 
    iteracción un usuario que cumpla los requisitos, si dicho usuario no 
    los cumple pasa al siguiente, pero a diferencia de la función anterior,
    esta no se contaría como interacción. Cuando termina la lista de un 
    usuario pone terminado a True y pasa al segundo nivel.

\setlength{\parskip}{1ex}
      \textbf{Parameters}
      \vspace{-1ex}

      \begin{quote}
        \begin{Ventry}{xxxxxxxxxxxxxxx}

          \item[G]

          Grafo al que añadiremos los nodos.

            {\it (type=nx.Graph())}

          \item[row\_count]

          Número de columnas en el archivo del usuario en cuestión.

            {\it (type=int)}

          \item[usu]

          Usuario del cual leemos sus seguidos.

            {\it (type=str)}

          \item[checkAnimales]

          Para saber si esta activado o no el checkBox de animales.

            {\it (type=boolean)}

          \item[checkRopa]

          Para saber si esta activado o no el checkBox de ropa.

            {\it (type=boolean)}

          \item[checkTerrorismo]

          Para saber si esta activado o no el checkBox de terrorismo.

            {\it (type=boolean)}

          \item[checkComida]

          Para saber si esta activado o no el checkBox de comida.

            {\it (type=boolean)}

          \item[checkSc]

          Para saber si esta activado o no el checkBox de sin calificar.

            {\it (type=boolean)}

        \end{Ventry}

      \end{quote}

    \end{boxedminipage}

    \label{grafico:Grafico:sacarSeguidoresSegundoNivelAlgunasCategorias}
    \index{grafico \textit{(module)}!grafico.Grafico \textit{(class)}!grafico.Grafico.sacarSeguidoresSegundoNivelAlgunasCategorias \textit{(method)}}

    \vspace{0.5ex}

\hspace{.8\funcindent}\begin{boxedminipage}{\funcwidth}

    \raggedright \textbf{sacarSeguidoresSegundoNivelAlgunasCategorias}(\textit{self}, \textit{usuario}, \textit{G})

    \vspace{-1.5ex}

    \rule{\textwidth}{0.5\fboxrule}
\setlength{\parskip}{2ex}
    Leemmos la lista de nodos que hay en G. Por cada iteracción iremos 
    borrando de la lista el nodo consultado y sacando sus seguidos para 
    posterior estudio.

\setlength{\parskip}{1ex}
      \textbf{Parameters}
      \vspace{-1ex}

      \begin{quote}
        \begin{Ventry}{xxxxxxx}

          \item[usuario]

          Usuario a partir del cual leemos sus seguidos, y de estos sacamos
          sus seguidos.

            {\it (type=str)}

          \item[G]

          Grafo al que consultamos sus nodos.

            {\it (type=nx.Graph())}

        \end{Ventry}

      \end{quote}

    \end{boxedminipage}

    \label{grafico:Grafico:colorearNodos}
    \index{grafico \textit{(module)}!grafico.Grafico \textit{(class)}!grafico.Grafico.colorearNodos \textit{(method)}}

    \vspace{0.5ex}

\hspace{.8\funcindent}\begin{boxedminipage}{\funcwidth}

    \raggedright \textbf{colorearNodos}(\textit{self}, \textit{G}, \textit{color\_map})

    \vspace{-1.5ex}

    \rule{\textwidth}{0.5\fboxrule}
\setlength{\parskip}{2ex}
    Se colorean los nodos en función del valor de los contoadores de cada 
    usuario. Los colores serían:

    \begin{itemize}
    \setlength{\parskip}{0.6ex}
      \item \textbf{Gris}: Sin calificar.

      \item \textbf{Rojo}: Terrorismo.

      \item \textbf{Verde}: Animales.

      \item \textbf{Azul}: Comida.

      \item \textbf{Amarillo}: Ropa.

    \end{itemize}

\setlength{\parskip}{1ex}
      \textbf{Parameters}
      \vspace{-1ex}

      \begin{quote}
        \begin{Ventry}{xxxxxxxxx}

          \item[G]

          Grafo al que consultamos sus nodos.

            {\it (type=nx.Graph())}

          \item[color\_map]

          Lista para colorear los nodos.

            {\it (type=list)}

        \end{Ventry}

      \end{quote}

    \end{boxedminipage}

    \label{grafico:Grafico:borrarGrafo}
    \index{grafico \textit{(module)}!grafico.Grafico \textit{(class)}!grafico.Grafico.borrarGrafo \textit{(method)}}

    \vspace{0.5ex}

\hspace{.8\funcindent}\begin{boxedminipage}{\funcwidth}

    \raggedright \textbf{borrarGrafo}(\textit{self})

    \vspace{-1.5ex}

    \rule{\textwidth}{0.5\fboxrule}
\setlength{\parskip}{2ex}
    Borra el Grafo.

\setlength{\parskip}{1ex}
    \end{boxedminipage}

    \label{grafico:Grafico:definirRowCount}
    \index{grafico \textit{(module)}!grafico.Grafico \textit{(class)}!grafico.Grafico.definirRowCount \textit{(method)}}

    \vspace{0.5ex}

\hspace{.8\funcindent}\begin{boxedminipage}{\funcwidth}

    \raggedright \textbf{definirRowCount}(\textit{self}, \textit{usu})

    \vspace{-1.5ex}

    \rule{\textwidth}{0.5\fboxrule}
\setlength{\parskip}{2ex}
    Define el número de columnas que hay en el archivo de seguidos del 
    usuario \textbf{usu}.

\setlength{\parskip}{1ex}
      \textbf{Parameters}
      \vspace{-1ex}

      \begin{quote}
        \begin{Ventry}{xxx}

          \item[usu]

          Usuario al que queremos sacar las columnas de su archivo de 
          seguidos.

            {\it (type=str)}

        \end{Ventry}

      \end{quote}

      \textbf{Return Value}
    \vspace{-1ex}

      \begin{quote}
      row\_count.

      {\it (type=int)}

      \end{quote}

    \end{boxedminipage}

    \label{grafico:Grafico:terminar}
    \index{grafico \textit{(module)}!grafico.Grafico \textit{(class)}!grafico.Grafico.terminar \textit{(method)}}

    \vspace{0.5ex}

\hspace{.8\funcindent}\begin{boxedminipage}{\funcwidth}

    \raggedright \textbf{terminar}(\textit{self})

    \vspace{-1.5ex}

    \rule{\textwidth}{0.5\fboxrule}
\setlength{\parskip}{2ex}
    Bloquea los botones al terminar el Grafo por completo.

\setlength{\parskip}{1ex}
    \end{boxedminipage}

    \label{grafico:Grafico:errorSinCategoria}
    \index{grafico \textit{(module)}!grafico.Grafico \textit{(class)}!grafico.Grafico.errorSinCategoria \textit{(method)}}

    \vspace{0.5ex}

\hspace{.8\funcindent}\begin{boxedminipage}{\funcwidth}

    \raggedright \textbf{errorSinCategoria}(\textit{self})

    \vspace{-1.5ex}

    \rule{\textwidth}{0.5\fboxrule}
\setlength{\parskip}{2ex}
    Muestra un error en caso de que no se seleccione ninguna categoría.

\setlength{\parskip}{1ex}
    \end{boxedminipage}

    \label{grafico:Grafico:dibujar}
    \index{grafico \textit{(module)}!grafico.Grafico \textit{(class)}!grafico.Grafico.dibujar \textit{(method)}}

    \vspace{0.5ex}

\hspace{.8\funcindent}\begin{boxedminipage}{\funcwidth}

    \raggedright \textbf{dibujar}(\textit{self}, \textit{checkAnimales}, \textit{checkRopa}, \textit{checkTerrorismo}, \textit{checkComida}, \textit{checkSc}, \textit{manejoGrafico})

    \vspace{-1.5ex}

    \rule{\textwidth}{0.5\fboxrule}
\setlength{\parskip}{2ex}
    Esta función se divide en tres partes. La primera, si todos los 
    checkBox están activados, en tal caso se llaman a las funciones 
    \textit{leerDocumentoTodasCategorias} y 
    \textit{sacarSeguidoresSegundoNivel} para el correcto funcionamiento 
    del grafo. La segunda, no todos los checkBox están activados, en tal 
    caso se llaman a las funciones \textit{leerDocumentoAlgunasCategorias} 
    y \textit{sacarSeguidoresSegundoNivelAlgunasCategorias} para el 
    correcto funcionamiento del grafo. Y por último, si no hay ningún 
    checkBox activado, en tal caso se procede al error.

\setlength{\parskip}{1ex}
      \textbf{Parameters}
      \vspace{-1ex}

      \begin{quote}
        \begin{Ventry}{xxxxxxxxxxxxxxx}

          \item[checkAnimales]

          Para saber si esta activado o no el checkBox de animales.

            {\it (type=boolean)}

          \item[checkRopa]

          Para saber si esta activado o no el checkBox de ropa.

            {\it (type=boolean)}

          \item[checkTerrorismo]

          Para saber si esta activado o no el checkBox de terrorismo.

            {\it (type=boolean)}

          \item[checkComida]

          Para saber si esta activado o no el checkBox de comida.

            {\it (type=boolean)}

          \item[checkSc]

          Para saber si esta activado o no el checkBox de sin calificar.

            {\it (type=boolean)}

          \item[manejoGrafico]

          Ventana de manejoGrafico para manejar el comportamiento de sus 
          componentes.

            {\it (type=object)}

        \end{Ventry}

      \end{quote}

    \end{boxedminipage}

    \index{grafico \textit{(module)}!grafico.Grafico \textit{(class)}|)}
    \index{grafico \textit{(module)}|)}
