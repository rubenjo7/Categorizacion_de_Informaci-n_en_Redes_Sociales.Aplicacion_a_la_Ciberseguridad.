%
% API Documentation for Categorizacion de Informacion en Redes Sociales. Aplicacion a la Ciberseguridad.
% Module informacion
%
% Generated by epydoc 3.0.1
% [Sat Jun 10 22:41:26 2017]
%

%%%%%%%%%%%%%%%%%%%%%%%%%%%%%%%%%%%%%%%%%%%%%%%%%%%%%%%%%%%%%%%%%%%%%%%%%%%
%%                          Module Description                           %%
%%%%%%%%%%%%%%%%%%%%%%%%%%%%%%%%%%%%%%%%%%%%%%%%%%%%%%%%%%%%%%%%%%%%%%%%%%%

    \index{informacion \textit{(module)}|(}
\section{Module informacion}

    \label{informacion}
Este módulo contiene la clase \textbf{Informacion}.

\textbf{Author:} Rubén Jiménez Ortega.



\textbf{Version:} 1.0



\textbf{Copyright:} Copyright (C) 2017 by Rubén Jiménez Ortega.




%%%%%%%%%%%%%%%%%%%%%%%%%%%%%%%%%%%%%%%%%%%%%%%%%%%%%%%%%%%%%%%%%%%%%%%%%%%
%%                           Class Description                           %%
%%%%%%%%%%%%%%%%%%%%%%%%%%%%%%%%%%%%%%%%%%%%%%%%%%%%%%%%%%%%%%%%%%%%%%%%%%%

    \index{informacion \textit{(module)}!informacion.Informacion \textit{(class)}|(}
\subsection{Class Informacion}

    \label{informacion:Informacion}
\begin{tabular}{cccccc}
% Line for object, linespec=[False]
\multicolumn{2}{r}{\settowidth{\BCL}{object}\multirow{2}{\BCL}{object}}
&&
  \\\cline{3-3}
  &&\multicolumn{1}{c|}{}
&&
  \\
&&\multicolumn{2}{l}{\textbf{informacion.Informacion}}
\end{tabular}

Crea un object para poder manejar la información del usuario.


%%%%%%%%%%%%%%%%%%%%%%%%%%%%%%%%%%%%%%%%%%%%%%%%%%%%%%%%%%%%%%%%%%%%%%%%%%%
%%                                Methods                                %%
%%%%%%%%%%%%%%%%%%%%%%%%%%%%%%%%%%%%%%%%%%%%%%%%%%%%%%%%%%%%%%%%%%%%%%%%%%%

  \subsubsection{Methods}

    \label{informacion:Informacion:setupUi}
    \index{informacion \textit{(module)}!informacion.Informacion \textit{(class)}!informacion.Informacion.setupUi \textit{(method)}}

    \vspace{0.5ex}

\hspace{.8\funcindent}\begin{boxedminipage}{\funcwidth}

    \raggedright \textbf{setupUi}(\textit{self}, \textit{Form})

    \vspace{-1.5ex}

    \rule{\textwidth}{0.5\fboxrule}
\setlength{\parskip}{2ex}
    Actualiza la ventana que se le pasa por parámetro añadiendole los 
    componentes.

\setlength{\parskip}{1ex}
      \textbf{Parameters}
      \vspace{-1ex}

      \begin{quote}
        \begin{Ventry}{xxxx}

          \item[Form]

          Ventana.

        \end{Ventry}

      \end{quote}

    \end{boxedminipage}

    \label{informacion:Informacion:retranslateUi}
    \index{informacion \textit{(module)}!informacion.Informacion \textit{(class)}!informacion.Informacion.retranslateUi \textit{(method)}}

    \vspace{0.5ex}

\hspace{.8\funcindent}\begin{boxedminipage}{\funcwidth}

    \raggedright \textbf{retranslateUi}(\textit{self}, \textit{Form}, \textit{nombre}, \textit{alias}, \textit{ide}, \textit{descripcion}, \textit{ubicacion}, \textit{seguidores}, \textit{siguiendo}, \textit{tweets}, \textit{fav}, \textit{foto}, \textit{comida}, \textit{animales}, \textit{ropa}, \textit{terrorismo}, \textit{Sc}, \textit{privacidad})

    \vspace{-1.5ex}

    \rule{\textwidth}{0.5\fboxrule}
\setlength{\parskip}{2ex}
    Actualiza la ventana que se le pasa por parámetro añadiendole los 
    verdaderos valores de los componentes.

\setlength{\parskip}{1ex}
      \textbf{Parameters}
      \vspace{-1ex}

      \begin{quote}
        \begin{Ventry}{xxxxxxxxxxx}

          \item[Form]

          Ventana

          \item[nombre]

          contiene el nombre completo del usuario.

            {\it (type=str)}

          \item[alias]

          contiene el nombre de usuario.

            {\it (type=str)}

          \item[ide]

          Contiene el identificador del usuario.

            {\it (type=int)}

          \item[descripcion]

          Contiene la descripción del usuario.

            {\it (type=str)}

          \item[ubicacion]

          Contiene la localización del usuario.

            {\it (type=str)}

          \item[seguidores]

          Contiene el número de seguidores del usuario.

            {\it (type=int)}

          \item[siguiendo]

          Contiene el número de seguidos del usuario.

            {\it (type=int)}

          \item[tweets]

          Contiene el número de tweets del usuario.

            {\it (type=int)}

          \item[fav]

          Contiene el número de tweets favoritos del usuario.

            {\it (type=int)}

          \item[foto]

          Contiene la dirección de la foto de usuario.

            {\it (type=str)}

          \item[comida]

          Contiene el contador de la categoría comida.

            {\it (type=int)}

          \item[animales]

          Contiene el contador de la categoría animales.

            {\it (type=int)}

          \item[ropa]

          Contiene el contador de la categoría ropa.

            {\it (type=int)}

          \item[terrorismo]

          Contiene el contador de la categoría terrorismo.

            {\it (type=int)}

          \item[Sc]

          Contiene el contador de la categoría sinCalificar.

            {\it (type=int)}

          \item[privacidad]

          Contiene si el usuario es privado o no.

            {\it (type=boolean)}

        \end{Ventry}

      \end{quote}

    \end{boxedminipage}

    \label{informacion:Informacion:pulsarComida}
    \index{informacion \textit{(module)}!informacion.Informacion \textit{(class)}!informacion.Informacion.pulsarComida \textit{(method)}}

    \vspace{0.5ex}

\hspace{.8\funcindent}\begin{boxedminipage}{\funcwidth}

    \raggedright \textbf{pulsarComida}(\textit{self}, \textit{event})

    \vspace{-1.5ex}

    \rule{\textwidth}{0.5\fboxrule}
\setlength{\parskip}{2ex}
    Abre la ventana categoría comida.

\setlength{\parskip}{1ex}
      \textbf{Parameters}
      \vspace{-1ex}

      \begin{quote}
        \begin{Ventry}{xxxxx}

          \item[event]

          evento.

        \end{Ventry}

      \end{quote}

    \end{boxedminipage}

    \label{informacion:Informacion:pulsarAnimales}
    \index{informacion \textit{(module)}!informacion.Informacion \textit{(class)}!informacion.Informacion.pulsarAnimales \textit{(method)}}

    \vspace{0.5ex}

\hspace{.8\funcindent}\begin{boxedminipage}{\funcwidth}

    \raggedright \textbf{pulsarAnimales}(\textit{self}, \textit{event})

    \vspace{-1.5ex}

    \rule{\textwidth}{0.5\fboxrule}
\setlength{\parskip}{2ex}
    Abre la ventana categoría animales.

\setlength{\parskip}{1ex}
      \textbf{Parameters}
      \vspace{-1ex}

      \begin{quote}
        \begin{Ventry}{xxxxx}

          \item[event]

          evento.

        \end{Ventry}

      \end{quote}

    \end{boxedminipage}

    \label{informacion:Informacion:pulsarRopa}
    \index{informacion \textit{(module)}!informacion.Informacion \textit{(class)}!informacion.Informacion.pulsarRopa \textit{(method)}}

    \vspace{0.5ex}

\hspace{.8\funcindent}\begin{boxedminipage}{\funcwidth}

    \raggedright \textbf{pulsarRopa}(\textit{self}, \textit{event})

    \vspace{-1.5ex}

    \rule{\textwidth}{0.5\fboxrule}
\setlength{\parskip}{2ex}
    Abre la ventana categoría ropa.

\setlength{\parskip}{1ex}
      \textbf{Parameters}
      \vspace{-1ex}

      \begin{quote}
        \begin{Ventry}{xxxxx}

          \item[event]

          evento.

        \end{Ventry}

      \end{quote}

    \end{boxedminipage}

    \label{informacion:Informacion:pulsarTerrorismo}
    \index{informacion \textit{(module)}!informacion.Informacion \textit{(class)}!informacion.Informacion.pulsarTerrorismo \textit{(method)}}

    \vspace{0.5ex}

\hspace{.8\funcindent}\begin{boxedminipage}{\funcwidth}

    \raggedright \textbf{pulsarTerrorismo}(\textit{self}, \textit{event})

    \vspace{-1.5ex}

    \rule{\textwidth}{0.5\fboxrule}
\setlength{\parskip}{2ex}
    Abre la ventana categoría terrorismo.

\setlength{\parskip}{1ex}
      \textbf{Parameters}
      \vspace{-1ex}

      \begin{quote}
        \begin{Ventry}{xxxxx}

          \item[event]

          evento.

        \end{Ventry}

      \end{quote}

    \end{boxedminipage}

    \label{informacion:Informacion:pulsarSc}
    \index{informacion \textit{(module)}!informacion.Informacion \textit{(class)}!informacion.Informacion.pulsarSc \textit{(method)}}

    \vspace{0.5ex}

\hspace{.8\funcindent}\begin{boxedminipage}{\funcwidth}

    \raggedright \textbf{pulsarSc}(\textit{self}, \textit{event})

    \vspace{-1.5ex}

    \rule{\textwidth}{0.5\fboxrule}
\setlength{\parskip}{2ex}
    Abre la ventana categoría sin calificar.

\setlength{\parskip}{1ex}
      \textbf{Parameters}
      \vspace{-1ex}

      \begin{quote}
        \begin{Ventry}{xxxxx}

          \item[event]

          evento.

        \end{Ventry}

      \end{quote}

    \end{boxedminipage}


\large{\textbf{\textit{Inherited from object}}}

\begin{quote}
\_\_delattr\_\_(), \_\_format\_\_(), \_\_getattribute\_\_(), \_\_hash\_\_(), \_\_init\_\_(), \_\_new\_\_(), \_\_reduce\_\_(), \_\_reduce\_ex\_\_(), \_\_repr\_\_(), \_\_setattr\_\_(), \_\_sizeof\_\_(), \_\_str\_\_(), \_\_subclasshook\_\_()
\end{quote}

%%%%%%%%%%%%%%%%%%%%%%%%%%%%%%%%%%%%%%%%%%%%%%%%%%%%%%%%%%%%%%%%%%%%%%%%%%%
%%                              Properties                               %%
%%%%%%%%%%%%%%%%%%%%%%%%%%%%%%%%%%%%%%%%%%%%%%%%%%%%%%%%%%%%%%%%%%%%%%%%%%%

  \subsubsection{Properties}

    \vspace{-1cm}
\hspace{\varindent}\begin{longtable}{|p{\varnamewidth}|p{\vardescrwidth}|l}
\cline{1-2}
\cline{1-2} \centering \textbf{Name} & \centering \textbf{Description}& \\
\cline{1-2}
\endhead\cline{1-2}\multicolumn{3}{r}{\small\textit{continued on next page}}\\\endfoot\cline{1-2}
\endlastfoot\multicolumn{2}{|l|}{\textit{Inherited from object}}\\
\multicolumn{2}{|p{\varwidth}|}{\raggedright \_\_class\_\_}\\
\cline{1-2}
\end{longtable}

    \index{informacion \textit{(module)}!informacion.Informacion \textit{(class)}|)}
    \index{informacion \textit{(module)}|)}
