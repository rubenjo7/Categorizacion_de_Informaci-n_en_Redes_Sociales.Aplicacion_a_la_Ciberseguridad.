%
% API Documentation for Categorizacion de Informacion en Redes Sociales. Aplicacion a la Ciberseguridad.
% Module creaVentanas
%
% Generated by epydoc 3.0.1
% [Sat Jun 10 22:41:26 2017]
%

%%%%%%%%%%%%%%%%%%%%%%%%%%%%%%%%%%%%%%%%%%%%%%%%%%%%%%%%%%%%%%%%%%%%%%%%%%%
%%                          Module Description                           %%
%%%%%%%%%%%%%%%%%%%%%%%%%%%%%%%%%%%%%%%%%%%%%%%%%%%%%%%%%%%%%%%%%%%%%%%%%%%

    \index{creaVentanas \textit{(module)}|(}
\section{Module creaVentanas}

    \label{creaVentanas}
Este módulo contiene la clase \textbf{CreaVentanas}.

\textbf{Author:} Rubén Jiménez Ortega.



\textbf{Version:} 1.0



\textbf{Copyright:} Copyright (C) 2017 by Rubén Jiménez Ortega.




%%%%%%%%%%%%%%%%%%%%%%%%%%%%%%%%%%%%%%%%%%%%%%%%%%%%%%%%%%%%%%%%%%%%%%%%%%%
%%                           Class Description                           %%
%%%%%%%%%%%%%%%%%%%%%%%%%%%%%%%%%%%%%%%%%%%%%%%%%%%%%%%%%%%%%%%%%%%%%%%%%%%

    \index{creaVentanas \textit{(module)}!creaVentanas.CreaVentanas \textit{(class)}|(}
\subsection{Class CreaVentanas}

    \label{creaVentanas:CreaVentanas}
Clase para crear la mayoría de las ventanas de las que dispone la 
aplicación.


%%%%%%%%%%%%%%%%%%%%%%%%%%%%%%%%%%%%%%%%%%%%%%%%%%%%%%%%%%%%%%%%%%%%%%%%%%%
%%                                Methods                                %%
%%%%%%%%%%%%%%%%%%%%%%%%%%%%%%%%%%%%%%%%%%%%%%%%%%%%%%%%%%%%%%%%%%%%%%%%%%%

  \subsubsection{Methods}

    \label{creaVentanas:CreaVentanas:center}
    \index{creaVentanas \textit{(module)}!creaVentanas.CreaVentanas \textit{(class)}!creaVentanas.CreaVentanas.center \textit{(method)}}

    \vspace{0.5ex}

\hspace{.8\funcindent}\begin{boxedminipage}{\funcwidth}

    \raggedright \textbf{center}(\textit{self}, \textit{qtgui})

    \vspace{-1.5ex}

    \rule{\textwidth}{0.5\fboxrule}
\setlength{\parskip}{2ex}
    Función para centrar cualquier ventana en una pantalla, 
    independientemente del tamaño de la misma.

\setlength{\parskip}{1ex}
      \textbf{Parameters}
      \vspace{-1ex}

      \begin{quote}
        \begin{Ventry}{xxxxx}

          \item[qtgui]

          Ventana que queremos centrar.

        \end{Ventry}

      \end{quote}

    \end{boxedminipage}

    \label{creaVentanas:CreaVentanas:crearVentanaInformacion}
    \index{creaVentanas \textit{(module)}!creaVentanas.CreaVentanas \textit{(class)}!creaVentanas.CreaVentanas.crearVentanaInformacion \textit{(method)}}

    \vspace{0.5ex}

\hspace{.8\funcindent}\begin{boxedminipage}{\funcwidth}

    \raggedright \textbf{crearVentanaInformacion}(\textit{self}, \textit{nombreUsuario}, \textit{usu}, \textit{ide}, \textit{descripcion}, \textit{localizacion}, \textit{nSeguidores}, \textit{nSeguidos}, \textit{nTweets}, \textit{nFavoritos}, \textit{fotoUsuario}, \textit{comida}, \textit{animales}, \textit{ropa}, \textit{terrorismo}, \textit{sinCalificar}, \textit{privacidad})

    \vspace{-1.5ex}

    \rule{\textwidth}{0.5\fboxrule}
\setlength{\parskip}{2ex}
    Crea la ventana que contiene la información de un usuario.

\setlength{\parskip}{1ex}
      \textbf{Parameters}
      \vspace{-1ex}

      \begin{quote}
        \begin{Ventry}{xxxxxxxxxxxxx}

          \item[nombreUsuario]

          contiene el nombre completo del usuario.

            {\it (type=str)}

          \item[usu]

          contiene el nombre de usuario.

            {\it (type=str)}

          \item[ide]

          Contiene el identificador del usuario.

            {\it (type=int)}

          \item[descripcion]

          Contiene la descripción del usuario.

            {\it (type=str)}

          \item[localizacion]

          Contiene la localización del usuario.

            {\it (type=str)}

          \item[nSeguidores]

          Contiene el número de seguidores del usuario.

            {\it (type=int)}

          \item[nSeguidos]

          Contiene el número de seguidos del usuario.

            {\it (type=int)}

          \item[nTweets]

          Contiene el número de tweets del usuario.

            {\it (type=int)}

          \item[nFavoritos]

          Contiene el número de tweets favoritos del usuario.

            {\it (type=int)}

          \item[fotoUsuario]

          Contiene la dirección de la foto de usuario.

            {\it (type=str)}

          \item[comida]

          Contiene el contador de la categoría comida.

            {\it (type=int)}

          \item[animales]

          Contiene el contador de la categoría animales.

            {\it (type=int)}

          \item[ropa]

          Contiene el contador de la categoría ropa.

            {\it (type=int)}

          \item[terrorismo]

          Contiene el contador de la categoría terrorismo.

            {\it (type=int)}

          \item[sinCalificar]

          Contiene el contador de la categoría sinCalificar.

            {\it (type=int)}

          \item[privacidad]

          Contiene si el usuario es privado o no.

            {\it (type=boolean)}

        \end{Ventry}

      \end{quote}

    \end{boxedminipage}

    \label{creaVentanas:CreaVentanas:crearVentanaManejoGrafica}
    \index{creaVentanas \textit{(module)}!creaVentanas.CreaVentanas \textit{(class)}!creaVentanas.CreaVentanas.crearVentanaManejoGrafica \textit{(method)}}

    \vspace{0.5ex}

\hspace{.8\funcindent}\begin{boxedminipage}{\funcwidth}

    \raggedright \textbf{crearVentanaManejoGrafica}(\textit{self})

    \vspace{-1.5ex}

    \rule{\textwidth}{0.5\fboxrule}
\setlength{\parskip}{2ex}
    Crea la ventana que maneja el Grafo.

\setlength{\parskip}{1ex}
    \end{boxedminipage}

    \label{creaVentanas:CreaVentanas:crearVentanaCategoria}
    \index{creaVentanas \textit{(module)}!creaVentanas.CreaVentanas \textit{(class)}!creaVentanas.CreaVentanas.crearVentanaCategoria \textit{(method)}}

    \vspace{0.5ex}

\hspace{.8\funcindent}\begin{boxedminipage}{\funcwidth}

    \raggedright \textbf{crearVentanaCategoria}(\textit{self}, \textit{titulo}, \textit{cat}, \textit{contador}, \textit{tweets})

    \vspace{-1.5ex}

    \rule{\textwidth}{0.5\fboxrule}
\setlength{\parskip}{2ex}
    Crea la ventana que contiene los tweets de una categoría.

\setlength{\parskip}{1ex}
      \textbf{Parameters}
      \vspace{-1ex}

      \begin{quote}
        \begin{Ventry}{xxxxxxxx}

          \item[titulo]

          Variable para cambiar el título de la nueva ventana.

            {\it (type=str)}

          \item[cat]

          Variable que almacena la categoría que mostramos.

            {\it (type=str)}

          \item[contador]

          Número de tweets de la categoría que estudiamos.

            {\it (type=int)}

          \item[tweets]

          Tweets del usuario y de la categoría elegida almacenados en una 
          lista para poder mostrarlos.

            {\it (type=list)}

        \end{Ventry}

      \end{quote}

    \end{boxedminipage}

    \label{creaVentanas:CreaVentanas:crearVentanaError}
    \index{creaVentanas \textit{(module)}!creaVentanas.CreaVentanas \textit{(class)}!creaVentanas.CreaVentanas.crearVentanaError \textit{(method)}}

    \vspace{0.5ex}

\hspace{.8\funcindent}\begin{boxedminipage}{\funcwidth}

    \raggedright \textbf{crearVentanaError}(\textit{self}, \textit{error})

    \vspace{-1.5ex}

    \rule{\textwidth}{0.5\fboxrule}
\setlength{\parskip}{2ex}
    Crea la ventana que advierte de un error.

\setlength{\parskip}{1ex}
      \textbf{Parameters}
      \vspace{-1ex}

      \begin{quote}
        \begin{Ventry}{xxxxx}

          \item[error]

          Error que se produce para mostrarlo en más detalles.

            {\it (type=str)}

        \end{Ventry}

      \end{quote}

    \end{boxedminipage}

    \label{creaVentanas:CreaVentanas:crearVentanaAlertaUsuario}
    \index{creaVentanas \textit{(module)}!creaVentanas.CreaVentanas \textit{(class)}!creaVentanas.CreaVentanas.crearVentanaAlertaUsuario \textit{(method)}}

    \vspace{0.5ex}

\hspace{.8\funcindent}\begin{boxedminipage}{\funcwidth}

    \raggedright \textbf{crearVentanaAlertaUsuario}(\textit{self})

    \vspace{-1.5ex}

    \rule{\textwidth}{0.5\fboxrule}
\setlength{\parskip}{2ex}
    Crea la ventana para mostrar que el nodo Raíz a sido cambiado con 
    éxito.

\setlength{\parskip}{1ex}
    \end{boxedminipage}

    \label{creaVentanas:CreaVentanas:crearVentanaFinGrafo}
    \index{creaVentanas \textit{(module)}!creaVentanas.CreaVentanas \textit{(class)}!creaVentanas.CreaVentanas.crearVentanaFinGrafo \textit{(method)}}

    \vspace{0.5ex}

\hspace{.8\funcindent}\begin{boxedminipage}{\funcwidth}

    \raggedright \textbf{crearVentanaFinGrafo}(\textit{self})

    \vspace{-1.5ex}

    \rule{\textwidth}{0.5\fboxrule}
\setlength{\parskip}{2ex}
    Crea la ventana para mostrar que el grafo a finalizado con éxito.

\setlength{\parskip}{1ex}
    \end{boxedminipage}

    \index{creaVentanas \textit{(module)}!creaVentanas.CreaVentanas \textit{(class)}|)}
    \index{creaVentanas \textit{(module)}|)}
