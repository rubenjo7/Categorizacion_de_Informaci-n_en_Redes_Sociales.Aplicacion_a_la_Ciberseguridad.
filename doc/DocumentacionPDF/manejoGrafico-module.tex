%
% API Documentation for Categorizacion de Informacion en Redes Sociales. Aplicacion a la Ciberseguridad.
% Module manejoGrafico
%
% Generated by epydoc 3.0.1
% [Sat Jun  3 09:18:27 2017]
%

%%%%%%%%%%%%%%%%%%%%%%%%%%%%%%%%%%%%%%%%%%%%%%%%%%%%%%%%%%%%%%%%%%%%%%%%%%%
%%                          Module Description                           %%
%%%%%%%%%%%%%%%%%%%%%%%%%%%%%%%%%%%%%%%%%%%%%%%%%%%%%%%%%%%%%%%%%%%%%%%%%%%

    \index{manejoGrafico \textit{(module)}|(}
\section{Module manejoGrafico}

    \label{manejoGrafico}
Este módulo contiene la clase \textbf{ManejoGrafico}.

\textbf{Author:} Rubén Jiménez Ortega.



\textbf{Version:} 1.0



\textbf{Copyright:} Copyright (C) 2017 by Rubén Jiménez Ortega.




%%%%%%%%%%%%%%%%%%%%%%%%%%%%%%%%%%%%%%%%%%%%%%%%%%%%%%%%%%%%%%%%%%%%%%%%%%%
%%                           Class Description                           %%
%%%%%%%%%%%%%%%%%%%%%%%%%%%%%%%%%%%%%%%%%%%%%%%%%%%%%%%%%%%%%%%%%%%%%%%%%%%

    \index{manejoGrafico \textit{(module)}!manejoGrafico.ManejoGrafico \textit{(class)}|(}
\subsection{Class ManejoGrafico}

    \label{manejoGrafico:ManejoGrafico}
\begin{tabular}{cccccc}
% Line for object, linespec=[False]
\multicolumn{2}{r}{\settowidth{\BCL}{object}\multirow{2}{\BCL}{object}}
&&
  \\\cline{3-3}
  &&\multicolumn{1}{c|}{}
&&
  \\
&&\multicolumn{2}{l}{\textbf{manejoGrafico.ManejoGrafico}}
\end{tabular}

Crea un object para poder manejar el Grafo.


%%%%%%%%%%%%%%%%%%%%%%%%%%%%%%%%%%%%%%%%%%%%%%%%%%%%%%%%%%%%%%%%%%%%%%%%%%%
%%                                Methods                                %%
%%%%%%%%%%%%%%%%%%%%%%%%%%%%%%%%%%%%%%%%%%%%%%%%%%%%%%%%%%%%%%%%%%%%%%%%%%%

  \subsubsection{Methods}

    \label{manejoGrafico:ManejoGrafico:setupUi}
    \index{manejoGrafico \textit{(module)}!manejoGrafico.ManejoGrafico \textit{(class)}!manejoGrafico.ManejoGrafico.setupUi \textit{(method)}}

    \vspace{0.5ex}

\hspace{.8\funcindent}\begin{boxedminipage}{\funcwidth}

    \raggedright \textbf{setupUi}(\textit{self}, \textit{Form})

    \vspace{-1.5ex}

    \rule{\textwidth}{0.5\fboxrule}
\setlength{\parskip}{2ex}
    Actualiza la ventana que se le pasa por parámetro añadiendole los 
    componentes.

\setlength{\parskip}{1ex}
      \textbf{Parameters}
      \vspace{-1ex}

      \begin{quote}
        \begin{Ventry}{xxxx}

          \item[Form]

          Ventana.

        \end{Ventry}

      \end{quote}

    \end{boxedminipage}

    \label{manejoGrafico:ManejoGrafico:retranslateUi}
    \index{manejoGrafico \textit{(module)}!manejoGrafico.ManejoGrafico \textit{(class)}!manejoGrafico.ManejoGrafico.retranslateUi \textit{(method)}}

    \vspace{0.5ex}

\hspace{.8\funcindent}\begin{boxedminipage}{\funcwidth}

    \raggedright \textbf{retranslateUi}(\textit{self}, \textit{Form})

    \vspace{-1.5ex}

    \rule{\textwidth}{0.5\fboxrule}
\setlength{\parskip}{2ex}
    Actualiza la ventana que se le pasa por parámetro añadiendole los 
    verdaderos valores de los componentes.

\setlength{\parskip}{1ex}
      \textbf{Parameters}
      \vspace{-1ex}

      \begin{quote}
        \begin{Ventry}{xxxx}

          \item[Form]

          Ventana.

        \end{Ventry}

      \end{quote}

    \end{boxedminipage}

    \label{manejoGrafico:ManejoGrafico:bloquearBotonesTerminar}
    \index{manejoGrafico \textit{(module)}!manejoGrafico.ManejoGrafico \textit{(class)}!manejoGrafico.ManejoGrafico.bloquearBotonesTerminar \textit{(method)}}

    \vspace{0.5ex}

\hspace{.8\funcindent}\begin{boxedminipage}{\funcwidth}

    \raggedright \textbf{bloquearBotonesTerminar}(\textit{self})

    \vspace{-1.5ex}

    \rule{\textwidth}{0.5\fboxrule}
\setlength{\parskip}{2ex}
    Bloquea los botones necesarios tras la finalización del grafo.

\setlength{\parskip}{1ex}
    \end{boxedminipage}

    \label{manejoGrafico:ManejoGrafico:errorNoCategorias}
    \index{manejoGrafico \textit{(module)}!manejoGrafico.ManejoGrafico \textit{(class)}!manejoGrafico.ManejoGrafico.errorNoCategorias \textit{(method)}}

    \vspace{0.5ex}

\hspace{.8\funcindent}\begin{boxedminipage}{\funcwidth}

    \raggedright \textbf{errorNoCategorias}(\textit{self})

    \vspace{-1.5ex}

    \rule{\textwidth}{0.5\fboxrule}
\setlength{\parskip}{2ex}
    Muestra el error de que no hay categorías seleccionadas y además 
    desbloquea y bloquea lo que se estima necesario.

\setlength{\parskip}{1ex}
    \end{boxedminipage}

    \label{manejoGrafico:ManejoGrafico:comprobar}
    \index{manejoGrafico \textit{(module)}!manejoGrafico.ManejoGrafico \textit{(class)}!manejoGrafico.ManejoGrafico.comprobar \textit{(method)}}

    \vspace{0.5ex}

\hspace{.8\funcindent}\begin{boxedminipage}{\funcwidth}

    \raggedright \textbf{comprobar}(\textit{self}, \textit{cuadroTexto})

    \vspace{-1.5ex}

    \rule{\textwidth}{0.5\fboxrule}
\setlength{\parskip}{2ex}
    Esta función llama a \textit{usuarioEstudiado} de la clase 
    \{funcionesTwitter\} para ver si ha sido estudiado el usuario con 
    anterioridad.

\setlength{\parskip}{1ex}
      \textbf{Parameters}
      \vspace{-1ex}

      \begin{quote}
        \begin{Ventry}{xxxxxxxxxxx}

          \item[cuadroTexto]

          Cuadro donde se inserta el usuario.

            {\it (type=QLineEdit())}

        \end{Ventry}

      \end{quote}

    \end{boxedminipage}

    \label{manejoGrafico:ManejoGrafico:cambioNodo}
    \index{manejoGrafico \textit{(module)}!manejoGrafico.ManejoGrafico \textit{(class)}!manejoGrafico.ManejoGrafico.cambioNodo \textit{(method)}}

    \vspace{0.5ex}

\hspace{.8\funcindent}\begin{boxedminipage}{\funcwidth}

    \raggedright \textbf{cambioNodo}(\textit{self}, \textit{cuadroTexto})

    \vspace{-1.5ex}

    \rule{\textwidth}{0.5\fboxrule}
\setlength{\parskip}{2ex}
    Esta función llama a \textit{usuarioEstudiadoNodoRaiz} de la clase 
    \{funcionesTwitter\} para ver si ha sido estudiado el usuario con 
    anterioridad y poder cambiarlo a nodo raíz o no.

\setlength{\parskip}{1ex}
      \textbf{Parameters}
      \vspace{-1ex}

      \begin{quote}
        \begin{Ventry}{xxxxxxxxxxx}

          \item[cuadroTexto]

          Cuadro donde se inserta el usuario.

            {\it (type=QLineEdit())}

        \end{Ventry}

      \end{quote}

    \end{boxedminipage}

    \label{manejoGrafico:ManejoGrafico:salir}
    \index{manejoGrafico \textit{(module)}!manejoGrafico.ManejoGrafico \textit{(class)}!manejoGrafico.ManejoGrafico.salir \textit{(method)}}

    \vspace{0.5ex}

\hspace{.8\funcindent}\begin{boxedminipage}{\funcwidth}

    \raggedright \textbf{salir}(\textit{self})

    \vspace{-1.5ex}

    \rule{\textwidth}{0.5\fboxrule}
\setlength{\parskip}{2ex}
    Cierra el programa y borra todos los archivos que se han creado.

\setlength{\parskip}{1ex}
    \end{boxedminipage}

    \label{manejoGrafico:ManejoGrafico:inicio}
    \index{manejoGrafico \textit{(module)}!manejoGrafico.ManejoGrafico \textit{(class)}!manejoGrafico.ManejoGrafico.inicio \textit{(method)}}

    \vspace{0.5ex}

\hspace{.8\funcindent}\begin{boxedminipage}{\funcwidth}

    \raggedright \textbf{inicio}(\textit{self})

    \vspace{-1.5ex}

    \rule{\textwidth}{0.5\fboxrule}
\setlength{\parskip}{2ex}
    Da comienzo al grafo o lo continua si ha sido pausado, bloqueando los 
    componentes necesarios para su correcto uso.

\setlength{\parskip}{1ex}
    \end{boxedminipage}

    \label{manejoGrafico:ManejoGrafico:pausa}
    \index{manejoGrafico \textit{(module)}!manejoGrafico.ManejoGrafico \textit{(class)}!manejoGrafico.ManejoGrafico.pausa \textit{(method)}}

    \vspace{0.5ex}

\hspace{.8\funcindent}\begin{boxedminipage}{\funcwidth}

    \raggedright \textbf{pausa}(\textit{self})

    \vspace{-1.5ex}

    \rule{\textwidth}{0.5\fboxrule}
\setlength{\parskip}{2ex}
    Pausa el grafo y bloquea los componentes necesarios para su correcto 
    uso.

\setlength{\parskip}{1ex}
    \end{boxedminipage}

    \label{manejoGrafico:ManejoGrafico:parar}
    \index{manejoGrafico \textit{(module)}!manejoGrafico.ManejoGrafico \textit{(class)}!manejoGrafico.ManejoGrafico.parar \textit{(method)}}

    \vspace{0.5ex}

\hspace{.8\funcindent}\begin{boxedminipage}{\funcwidth}

    \raggedright \textbf{parar}(\textit{self})

    \vspace{-1.5ex}

    \rule{\textwidth}{0.5\fboxrule}
\setlength{\parskip}{2ex}
    Detiene y borra el grafo y bloquea los componentes necesarios para su 
    correcto uso.

\setlength{\parskip}{1ex}
    \end{boxedminipage}


\large{\textbf{\textit{Inherited from object}}}

\begin{quote}
\_\_delattr\_\_(), \_\_format\_\_(), \_\_getattribute\_\_(), \_\_hash\_\_(), \_\_init\_\_(), \_\_new\_\_(), \_\_reduce\_\_(), \_\_reduce\_ex\_\_(), \_\_repr\_\_(), \_\_setattr\_\_(), \_\_sizeof\_\_(), \_\_str\_\_(), \_\_subclasshook\_\_()
\end{quote}

%%%%%%%%%%%%%%%%%%%%%%%%%%%%%%%%%%%%%%%%%%%%%%%%%%%%%%%%%%%%%%%%%%%%%%%%%%%
%%                              Properties                               %%
%%%%%%%%%%%%%%%%%%%%%%%%%%%%%%%%%%%%%%%%%%%%%%%%%%%%%%%%%%%%%%%%%%%%%%%%%%%

  \subsubsection{Properties}

    \vspace{-1cm}
\hspace{\varindent}\begin{longtable}{|p{\varnamewidth}|p{\vardescrwidth}|l}
\cline{1-2}
\cline{1-2} \centering \textbf{Name} & \centering \textbf{Description}& \\
\cline{1-2}
\endhead\cline{1-2}\multicolumn{3}{r}{\small\textit{continued on next page}}\\\endfoot\cline{1-2}
\endlastfoot\multicolumn{2}{|l|}{\textit{Inherited from object}}\\
\multicolumn{2}{|p{\varwidth}|}{\raggedright \_\_class\_\_}\\
\cline{1-2}
\end{longtable}

    \index{manejoGrafico \textit{(module)}!manejoGrafico.ManejoGrafico \textit{(class)}|)}
    \index{manejoGrafico \textit{(module)}|)}
