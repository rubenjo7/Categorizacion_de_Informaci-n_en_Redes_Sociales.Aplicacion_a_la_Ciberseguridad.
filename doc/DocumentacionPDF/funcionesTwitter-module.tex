%
% API Documentation for Categorizacion de Informacion en Redes Sociales. Aplicacion a la Ciberseguridad.
% Module funcionesTwitter
%
% Generated by epydoc 3.0.1
% [Sat Jun 10 22:41:26 2017]
%

%%%%%%%%%%%%%%%%%%%%%%%%%%%%%%%%%%%%%%%%%%%%%%%%%%%%%%%%%%%%%%%%%%%%%%%%%%%
%%                          Module Description                           %%
%%%%%%%%%%%%%%%%%%%%%%%%%%%%%%%%%%%%%%%%%%%%%%%%%%%%%%%%%%%%%%%%%%%%%%%%%%%

    \index{funcionesTwitter \textit{(module)}|(}
\section{Module funcionesTwitter}

    \label{funcionesTwitter}
Este módulo contiene la clase \textbf{FuncionesTwitter}.

\textbf{Author:} Rubén Jiménez Ortega.



\textbf{Version:} 1.0



\textbf{Copyright:} Copyright (C) 2017 by Rubén Jiménez Ortega.




%%%%%%%%%%%%%%%%%%%%%%%%%%%%%%%%%%%%%%%%%%%%%%%%%%%%%%%%%%%%%%%%%%%%%%%%%%%
%%                               Variables                               %%
%%%%%%%%%%%%%%%%%%%%%%%%%%%%%%%%%%%%%%%%%%%%%%%%%%%%%%%%%%%%%%%%%%%%%%%%%%%

  \subsection{Variables}

    \vspace{-1cm}
\hspace{\varindent}\begin{longtable}{|p{\varnamewidth}|p{\vardescrwidth}|l}
\cline{1-2}
\cline{1-2} \centering \textbf{Name} & \centering \textbf{Description}& \\
\cline{1-2}
\endhead\cline{1-2}\multicolumn{3}{r}{\small\textit{continued on next page}}\\\endfoot\cline{1-2}
\endlastfoot\raggedright a\-r\-c\-h\-i\-v\-o\-\_\-c\-o\-n\-s\-u\-m\-e\-r\-\_\-k\-e\-y\- & \raggedright Archivo donde leemos la clave del consumidor.

\textbf{Value:} 
{\tt open("../variables/consumer\_key.txt", "r")}&\\
\cline{1-2}
\raggedright c\-o\-n\-s\-u\-m\-e\-r\-\_\-k\-e\-y\- & \raggedright Clave del consumidor.

\textbf{Value:} 
{\tt archivo\_consumer\_key.readline() [:-1]}&\\
\cline{1-2}
\raggedright a\-r\-c\-h\-i\-v\-o\-\_\-c\-o\-n\-s\-u\-m\-e\-r\-\_\-s\-e\-c\-r\-e\-t\- & \raggedright Archivo donde leemos la clave secreta del consumidor.

\textbf{Value:} 
{\tt open("../variables/consumer\_secret.txt", "r")}&\\
\cline{1-2}
\raggedright c\-o\-n\-s\-u\-m\-e\-r\-\_\-s\-e\-c\-r\-e\-t\- & \raggedright Clave secreta del consumidor.

\textbf{Value:} 
{\tt archivo\_consumer\_secret.readline() [:-1]}&\\
\cline{1-2}
\raggedright a\-r\-c\-h\-i\-v\-o\-\_\-a\-c\-c\-e\-s\-s\-\_\-t\-o\-k\-e\-n\- & \raggedright Archivo donde leemos el token de acceso.

\textbf{Value:} 
{\tt open("../variables/access\_token.txt", "r")}&\\
\cline{1-2}
\raggedright a\-c\-c\-e\-s\-s\-\_\-t\-o\-k\-e\-n\- & \raggedright Token de acceso.

\textbf{Value:} 
{\tt archivo\_access\_token.readline() [:-1]}&\\
\cline{1-2}
\raggedright a\-r\-c\-h\-i\-v\-o\-\_\-a\-c\-c\-e\-s\-s\-\_\-t\-o\-k\-e\-n\-\_\-s\-e\-c\-r\-e\-t\- & \raggedright Archivo donde leemos el token de acceso secreto.

\textbf{Value:} 
{\tt open("../variables/access\_token\_secret.txt", "r")}&\\
\cline{1-2}
\raggedright a\-c\-c\-e\-s\-s\-\_\-t\-o\-k\-e\-n\-\_\-s\-e\-c\-r\-e\-t\- & \raggedright Token de acceso secreto.

\textbf{Value:} 
{\tt archivo\_access\_token\_secret.readline() [:-1]}&\\
\cline{1-2}
\raggedright a\-u\-t\-h\- & \raggedright Creación de una instancia OAuthHandler.

\textbf{Value:} 
{\tt tweepy.OAuthHandler(consumer\_key, consumer\_secret)}&\\
\cline{1-2}
\raggedright a\-p\-i\- & \raggedright Creación de la interface, usando auth.

\textbf{Value:} 
{\tt tweepy.API(auth)}&\\
\cline{1-2}
\end{longtable}


%%%%%%%%%%%%%%%%%%%%%%%%%%%%%%%%%%%%%%%%%%%%%%%%%%%%%%%%%%%%%%%%%%%%%%%%%%%
%%                           Class Description                           %%
%%%%%%%%%%%%%%%%%%%%%%%%%%%%%%%%%%%%%%%%%%%%%%%%%%%%%%%%%%%%%%%%%%%%%%%%%%%

    \index{funcionesTwitter \textit{(module)}!funcionesTwitter.FuncionesTwitter \textit{(class)}|(}
\subsection{Class FuncionesTwitter}

    \label{funcionesTwitter:FuncionesTwitter}
Esta clase contiene todos los métodos relacionados con la API de Twitter 
(tweepy):

\begin{itemize}
\setlength{\parskip}{0.6ex}
  \item Comprobar si un usuario es privado.

  \item Escribe los datos de un usuario.

  \item Escribe los tweets de un usuario.

  \item Obtiene los tweets separados en las distintas categorías.

  \item Cuenta y ordena de mayor a menor los tweets de las distintas 
    categorías.

  \item Lee los distintos diccionarios que han sido creados.

\end{itemize}


%%%%%%%%%%%%%%%%%%%%%%%%%%%%%%%%%%%%%%%%%%%%%%%%%%%%%%%%%%%%%%%%%%%%%%%%%%%
%%                                Methods                                %%
%%%%%%%%%%%%%%%%%%%%%%%%%%%%%%%%%%%%%%%%%%%%%%%%%%%%%%%%%%%%%%%%%%%%%%%%%%%

  \subsubsection{Methods}

    \label{funcionesTwitter:FuncionesTwitter:esPrivado}
    \index{funcionesTwitter \textit{(module)}!funcionesTwitter.FuncionesTwitter \textit{(class)}!funcionesTwitter.FuncionesTwitter.esPrivado \textit{(method)}}

    \vspace{0.5ex}

\hspace{.8\funcindent}\begin{boxedminipage}{\funcwidth}

    \raggedright \textbf{esPrivado}(\textit{self}, \textit{usu})

    \vspace{-1.5ex}

    \rule{\textwidth}{0.5\fboxrule}
\setlength{\parskip}{2ex}
    Comprueba si el usuario pasado por parámetro es privado o no.

\setlength{\parskip}{1ex}
      \textbf{Parameters}
      \vspace{-1ex}

      \begin{quote}
        \begin{Ventry}{xxx}

          \item[usu]

          Usuario al que se somete la comprobación.

            {\it (type=str)}

        \end{Ventry}

      \end{quote}

      \textbf{Return Value}
    \vspace{-1ex}

      \begin{quote}
      Devuelve un booleano si el usuario es privado o no.

      {\it (type=boolean)}

      \end{quote}

    \end{boxedminipage}

    \label{funcionesTwitter:FuncionesTwitter:existeUsuario}
    \index{funcionesTwitter \textit{(module)}!funcionesTwitter.FuncionesTwitter \textit{(class)}!funcionesTwitter.FuncionesTwitter.existeUsuario \textit{(method)}}

    \vspace{0.5ex}

\hspace{.8\funcindent}\begin{boxedminipage}{\funcwidth}

    \raggedright \textbf{existeUsuario}(\textit{self}, \textit{usuario}, \textit{ventanaInicio})

    \vspace{-1.5ex}

    \rule{\textwidth}{0.5\fboxrule}
\setlength{\parskip}{2ex}
    Comprueba si el usuario pasado por parámetro es privado o no. En caso 
    de que sea privado nos salta una ventana de error avisando de este 
    hecho. En caso de que no sea privado, se da comienzo a la interfaz que 
    maneja el grafo, se obtienen los seguidos y los tweets de ese usuario. 
    Además se coloca este usuario como nodo raíz de nuestro grafo.

\setlength{\parskip}{1ex}
      \textbf{Parameters}
      \vspace{-1ex}

      \begin{quote}
        \begin{Ventry}{xxxxxxxxxxxxx}

          \item[usuario]

          Usuario al que se somete las operaciones anteriores.

            {\it (type=str)}

          \item[ventanaInicio]

          Ventana en la que hay que insertar el usuario, para poder ocultar
          dicha ventana.

            {\it (type=QWidget)}

        \end{Ventry}

      \end{quote}

      \textbf{Raises}
    \vspace{-1ex}

      \begin{quote}
        \begin{description}

          \item[\texttt{tweepy.TweepError}]

          En el caso de que se inserte un usuario que no exista.

        \end{description}

      \end{quote}

    \end{boxedminipage}

    \label{funcionesTwitter:FuncionesTwitter:usuarioEstudiado}
    \index{funcionesTwitter \textit{(module)}!funcionesTwitter.FuncionesTwitter \textit{(class)}!funcionesTwitter.FuncionesTwitter.usuarioEstudiado \textit{(method)}}

    \vspace{0.5ex}

\hspace{.8\funcindent}\begin{boxedminipage}{\funcwidth}

    \raggedright \textbf{usuarioEstudiado}(\textit{self}, \textit{usuario})

    \vspace{-1.5ex}

    \rule{\textwidth}{0.5\fboxrule}
\setlength{\parskip}{2ex}
    Comprueba si el usuario pasado por parámetro ha sido estudiado con 
    anterioridad, es decir, que haya sido pintado en el Grafo. Si ha sido 
    estudiado se obtienen los datos de este usuario, en caso contrario se 
    genera un error informando que este usuario no ha sido estudiado.

\setlength{\parskip}{1ex}
      \textbf{Parameters}
      \vspace{-1ex}

      \begin{quote}
        \begin{Ventry}{xxxxxxx}

          \item[usuario]

          Usuario al que se somete las operaciones anteriores.

            {\it (type=str)}

        \end{Ventry}

      \end{quote}

      \textbf{Raises}
    \vspace{-1ex}

      \begin{quote}
        \begin{description}

          \item[\texttt{tweepy.TweepError}]

          En el caso de que se inserte un usuario que no exista.

        \end{description}

      \end{quote}

    \end{boxedminipage}

    \label{funcionesTwitter:FuncionesTwitter:usuarioEstudiadoNodoRaiz}
    \index{funcionesTwitter \textit{(module)}!funcionesTwitter.FuncionesTwitter \textit{(class)}!funcionesTwitter.FuncionesTwitter.usuarioEstudiadoNodoRaiz \textit{(method)}}

    \vspace{0.5ex}

\hspace{.8\funcindent}\begin{boxedminipage}{\funcwidth}

    \raggedright \textbf{usuarioEstudiadoNodoRaiz}(\textit{self}, \textit{usuario})

    \vspace{-1.5ex}

    \rule{\textwidth}{0.5\fboxrule}
\setlength{\parskip}{2ex}
    Comprueba si el usuario pasado por parámetro es privado, en ese caso, 
    aparece una ventana indicando ese error, en caso contrario, comprobamos
    si hemos estudiado este usuario, si ha sido estudiado, comprobamos si 
    tenemos sus seguidos, si no los tenemos, los sacamos. Además, cambiamos
    el nodo Raíz a este usuario y sacamos una ventan alertando este suceso.
    Si no ha sido estudiado se informa de este error.

\setlength{\parskip}{1ex}
      \textbf{Parameters}
      \vspace{-1ex}

      \begin{quote}
        \begin{Ventry}{xxxxxxx}

          \item[usuario]

          Usuario al que se somete las operaciones anteriores.

            {\it (type=str)}

        \end{Ventry}

      \end{quote}

      \textbf{Raises}
    \vspace{-1ex}

      \begin{quote}
        \begin{description}

          \item[\texttt{tweepy.TweepError}]

          En el caso de que se inserte un usuario que no exista.

        \end{description}

      \end{quote}

    \end{boxedminipage}

    \label{funcionesTwitter:FuncionesTwitter:seguidos}
    \index{funcionesTwitter \textit{(module)}!funcionesTwitter.FuncionesTwitter \textit{(class)}!funcionesTwitter.FuncionesTwitter.seguidos \textit{(method)}}

    \vspace{0.5ex}

\hspace{.8\funcindent}\begin{boxedminipage}{\funcwidth}

    \raggedright \textbf{seguidos}(\textit{self}, \textit{usu})

    \vspace{-1.5ex}

    \rule{\textwidth}{0.5\fboxrule}
\setlength{\parskip}{2ex}
    Si el usuario no es privado, pasamos a a obtener sus últimos 5000 
    seguidos (Ya que la API de twitter, nos limita este aspecto, para 
    evitar algunos problemas), estos seguidos serán guardados en un archivo
    csv para su posterior uso. En caso de ser privado se crea el archivo 
    pero vacio, para tener presente su estudio.

\setlength{\parskip}{1ex}
      \textbf{Parameters}
      \vspace{-1ex}

      \begin{quote}
        \begin{Ventry}{xxx}

          \item[usu]

          Usuario al que se somete las operaciones anteriores.

            {\it (type=str)}

        \end{Ventry}

      \end{quote}

    \end{boxedminipage}

    \label{funcionesTwitter:FuncionesTwitter:comienzoGrafo}
    \index{funcionesTwitter \textit{(module)}!funcionesTwitter.FuncionesTwitter \textit{(class)}!funcionesTwitter.FuncionesTwitter.comienzoGrafo \textit{(method)}}

    \vspace{0.5ex}

\hspace{.8\funcindent}\begin{boxedminipage}{\funcwidth}

    \raggedright \textbf{comienzoGrafo}(\textit{self})

    \vspace{-1.5ex}

    \rule{\textwidth}{0.5\fboxrule}
\setlength{\parskip}{2ex}
    Inicia la ventana que hace posible el manejo del Grafo.

\setlength{\parskip}{1ex}
    \end{boxedminipage}

    \label{funcionesTwitter:FuncionesTwitter:obtenerDatos}
    \index{funcionesTwitter \textit{(module)}!funcionesTwitter.FuncionesTwitter \textit{(class)}!funcionesTwitter.FuncionesTwitter.obtenerDatos \textit{(method)}}

    \vspace{0.5ex}

\hspace{.8\funcindent}\begin{boxedminipage}{\funcwidth}

    \raggedright \textbf{obtenerDatos}(\textit{self}, \textit{usuario})

    \vspace{-1.5ex}

    \rule{\textwidth}{0.5\fboxrule}
\setlength{\parskip}{2ex}
    Saca la información de un usuario que ya haya sido estudiado (Nombre de
    usuario, descripción, localización, número de seguidores, número de 
    seguidos, número de tweets, número de tweets Favoritos, la foto de 
    perfil, el identificador del usuario, si el usuario es privado o no y 
    los contadores de tweets en las diferentes categorías) para mostrarla 
    en la ventana de Información.

\setlength{\parskip}{1ex}
      \textbf{Parameters}
      \vspace{-1ex}

      \begin{quote}
        \begin{Ventry}{xxxxxxx}

          \item[usuario]

          Usuario al que se somete las operaciones anteriores.

            {\it (type=str)}

        \end{Ventry}

      \end{quote}

    \end{boxedminipage}

    \label{funcionesTwitter:FuncionesTwitter:contadorTweetsPorCategorias}
    \index{funcionesTwitter \textit{(module)}!funcionesTwitter.FuncionesTwitter \textit{(class)}!funcionesTwitter.FuncionesTwitter.contadorTweetsPorCategorias \textit{(method)}}

    \vspace{0.5ex}

\hspace{.8\funcindent}\begin{boxedminipage}{\funcwidth}

    \raggedright \textbf{contadorTweetsPorCategorias}(\textit{self}, \textit{usu}, \textit{nTweets})

    \vspace{-1.5ex}

    \rule{\textwidth}{0.5\fboxrule}
\setlength{\parskip}{2ex}
    Esta función se encarga de organizar los tweets por categorías y sus 
    respectivos contadores siempre que el usuario no sea privado, en cuyo 
    caso sus contadores se pondrán a 0 por defecto. Decir también que la 
    limitación que nos da la API de twitter en este caso es que no se 
    podrán obtener más de los 3200 últimos tweets de cada usuario. Para 
    sacar la categoría en la que se clasificará cada tweet, se comprueba 
    los diccionarios que han sido creados con anteridad (estos se pueden 
    aumentar o variar siempre y cuando se crea necesario, editando sus 
    respectivos ficheros) y se compara cada palabra de esos diccionarios 
    con todos los tweets estudiados, si aparece una de las palabras de ese 
    diccionario en un tweet, este será clasificado directamente a esa 
    categoría. Tras esto se procede a escribir los contadores de las 
    categorías en un fichero, que será consultado si se desea ver la 
    información de un usuario. Y los tweets por categorías se escribirá un 
    fichero por cada categoría siempre y cuando el contador sea distinto de
    0.

\setlength{\parskip}{1ex}
      \textbf{Parameters}
      \vspace{-1ex}

      \begin{quote}
        \begin{Ventry}{xxxxxxx}

          \item[usu]

          Usuario al que se somete las operaciones anteriores.

            {\it (type=str)}

          \item[nTweets]

          Número totales de tweets del usuario estudiado.

            {\it (type=int)}

        \end{Ventry}

      \end{quote}

    \end{boxedminipage}

    \label{funcionesTwitter:FuncionesTwitter:escribirDatosUsuario}
    \index{funcionesTwitter \textit{(module)}!funcionesTwitter.FuncionesTwitter \textit{(class)}!funcionesTwitter.FuncionesTwitter.escribirDatosUsuario \textit{(method)}}

    \vspace{0.5ex}

\hspace{.8\funcindent}\begin{boxedminipage}{\funcwidth}

    \raggedright \textbf{escribirDatosUsuario}(\textit{self}, \textit{usuario}, \textit{contadorComida}, \textit{contadorRopa}, \textit{contadorAnimales}, \textit{contadorTerrorismo}, \textit{contadorSinCalificar})

    \vspace{-1.5ex}

    \rule{\textwidth}{0.5\fboxrule}
\setlength{\parskip}{2ex}
    Función para guardar los contadores de las categorías en un archivo que
    sigue la siguiente sintaxis: "usuario\_tweets.csv" siendo usuario uno 
    de los parámetros que se pasan a la función. Este archivo a parte de 
    los contadores, se almacena también el identificador del usuario y el 
    nombre del usuario.

\setlength{\parskip}{1ex}
      \textbf{Parameters}
      \vspace{-1ex}

      \begin{quote}
        \begin{Ventry}{xxxxxxxxxxxxxxxxxxxx}

          \item[usuario]

          Nombre del usuario para el nombre y el contenido del archivo 
          nuevo.

            {\it (type=str)}

          \item[contadorComida]

          Variable que contiene el contador de comida.

            {\it (type=int)}

          \item[contadorRopa]

          Variable que contiene el contador de ropa.

            {\it (type=int)}

          \item[contadorAnimales]

          Variable que contiene el contador de animales.

            {\it (type=int)}

          \item[contadorTerrorismo]

          Variable que contiene el contador de terrorismo.

            {\it (type=int)}

          \item[contadorSinCalificar]

          Variable que contiene el contador de de los tweets sin calificar.

            {\it (type=int)}

        \end{Ventry}

      \end{quote}

    \end{boxedminipage}

    \label{funcionesTwitter:FuncionesTwitter:escribirTweetsClasificados}
    \index{funcionesTwitter \textit{(module)}!funcionesTwitter.FuncionesTwitter \textit{(class)}!funcionesTwitter.FuncionesTwitter.escribirTweetsClasificados \textit{(method)}}

    \vspace{0.5ex}

\hspace{.8\funcindent}\begin{boxedminipage}{\funcwidth}

    \raggedright \textbf{escribirTweetsClasificados}(\textit{self}, \textit{usu}, \textit{tweetsComida}, \textit{tweetsRopa}, \textit{tweetsAnimales}, \textit{tweetsTerrorismo}, \textit{tweetsSc})

    \vspace{-1.5ex}

    \rule{\textwidth}{0.5\fboxrule}
\setlength{\parskip}{2ex}
    Función para guardar los tweets de cada categoría en su respectivo 
    archivo. Los tweets vienen en forma de lista y se pasan a los archivos 
    que siguen la siguiente sintaxis: "usu\_tweets\_*.csv", siendo "usu" el
    usuario pasado por parámetro y "*" las distintas categorías se pueden 
    dar.

\setlength{\parskip}{1ex}
      \textbf{Parameters}
      \vspace{-1ex}

      \begin{quote}
        \begin{Ventry}{xxxxxxxxxxxxxxxx}

          \item[usu]

          Nombre del usuario para el nombre de los archivos que se crean.

            {\it (type=str)}

          \item[tweetsComida]

          Variable que contiene la lista de de los tweets de comida.

            {\it (type=list)}

          \item[tweetsRopa]

          Variable que contiene la lista de de los tweets de ropa.

            {\it (type=list)}

          \item[tweetsAnimales]

          Variable que contiene la lista de de los tweets de animales.

            {\it (type=list)}

          \item[tweetsTerrorismo]

          Variable que contiene la lista de de los tweets de terrorismo.

            {\it (type=list)}

          \item[tweetsSc]

          Variable que contiene la lista de de los tweets sin calificar.

            {\it (type=list)}

        \end{Ventry}

      \end{quote}

    \end{boxedminipage}

    \label{funcionesTwitter:FuncionesTwitter:ordenarContadores}
    \index{funcionesTwitter \textit{(module)}!funcionesTwitter.FuncionesTwitter \textit{(class)}!funcionesTwitter.FuncionesTwitter.ordenarContadores \textit{(method)}}

    \vspace{0.5ex}

\hspace{.8\funcindent}\begin{boxedminipage}{\funcwidth}

    \raggedright \textbf{ordenarContadores}(\textit{self}, \textit{usuario})

    \vspace{-1.5ex}

    \rule{\textwidth}{0.5\fboxrule}
\setlength{\parskip}{2ex}
    Ordena los contadores de las distintas categorías de un usuario. La 
    prioridad que se sigue en caso de igualdad es la siguiente: terrorismo 
    - animales - comida - ropa, dejando la categoría sin calificar para 
    cuando se de el caso de que todas las demás categorías sean 0 o bien 
    cuando el usuario sea privado y no podamos acceder a sus tweets.

\setlength{\parskip}{1ex}
      \textbf{Parameters}
      \vspace{-1ex}

      \begin{quote}
        \begin{Ventry}{xxxxxxx}

          \item[usuario]

          Usuario al que queremos conocer su contador mayor.

            {\it (type=str)}

        \end{Ventry}

      \end{quote}

      \textbf{Return Value}
    \vspace{-1ex}

      \begin{quote}
      Devuelve \textbf{valor} en función de que categoría que tenga mayor 
      contador. Los valores que se pueden dar son los siguientes: Si 
      \textit{valor = 0} --{\textgreater} categoría sin calificar, si 
      \textit{valor = 1} --{\textgreater} categoría terrorismo, si 
      \textit{valor = 2} --{\textgreater} categoría animales, si 
      \textit{valor = 3} --{\textgreater} categoría comida, si 
      \textit{valor = 4} --{\textgreater} categoría Ropa.

      {\it (type=int)}

      \end{quote}

    \end{boxedminipage}

    \label{funcionesTwitter:FuncionesTwitter:leerContadores}
    \index{funcionesTwitter \textit{(module)}!funcionesTwitter.FuncionesTwitter \textit{(class)}!funcionesTwitter.FuncionesTwitter.leerContadores \textit{(method)}}

    \vspace{0.5ex}

\hspace{.8\funcindent}\begin{boxedminipage}{\funcwidth}

    \raggedright \textbf{leerContadores}(\textit{self}, \textit{usuario})

    \vspace{-1.5ex}

    \rule{\textwidth}{0.5\fboxrule}
\setlength{\parskip}{2ex}
    Esta función se encarga de leer los contadores que se encuentran 
    almacenados en los ficheros "usuario\_tweets.csv", siendo 
    \textit{usuario} el parámetro que pasamos a esta función.

\setlength{\parskip}{1ex}
      \textbf{Parameters}
      \vspace{-1ex}

      \begin{quote}
        \begin{Ventry}{xxxxxxx}

          \item[usuario]

          Usuario al que queremos conocer sus contadores.

            {\it (type=str)}

        \end{Ventry}

      \end{quote}

      \textbf{Return Value}
    \vspace{-1ex}

      \begin{quote}
      Devuelve los valores de los contadores de \textbf{comida}, 
      \textbf{ropa}, \textbf{animales}, \textbf{terrorismo} y 
      \textbf{sinCalificar}.

      {\it (type=int)}

      \end{quote}

    \end{boxedminipage}

    \label{funcionesTwitter:FuncionesTwitter:obtenerTweetsAnimales}
    \index{funcionesTwitter \textit{(module)}!funcionesTwitter.FuncionesTwitter \textit{(class)}!funcionesTwitter.FuncionesTwitter.obtenerTweetsAnimales \textit{(method)}}

    \vspace{0.5ex}

\hspace{.8\funcindent}\begin{boxedminipage}{\funcwidth}

    \raggedright \textbf{obtenerTweetsAnimales}(\textit{self}, \textit{usu})

    \vspace{-1.5ex}

    \rule{\textwidth}{0.5\fboxrule}
\setlength{\parskip}{2ex}
    Esta función se encarga de obtener los tweets relacionados con los 
    animales para el usuario \textbf{usu}. Para ello, se llamamos a la 
    función \textit{leerTweetsAnimales(usu)}, y una vez leidos, creamos la 
    ventana con estos tweets.

\setlength{\parskip}{1ex}
      \textbf{Parameters}
      \vspace{-1ex}

      \begin{quote}
        \begin{Ventry}{xxx}

          \item[usu]

          Usuario del que queremos conocer sus tweets de animales.

            {\it (type=str)}

        \end{Ventry}

      \end{quote}

    \end{boxedminipage}

    \label{funcionesTwitter:FuncionesTwitter:leerTweetsAnimales}
    \index{funcionesTwitter \textit{(module)}!funcionesTwitter.FuncionesTwitter \textit{(class)}!funcionesTwitter.FuncionesTwitter.leerTweetsAnimales \textit{(method)}}

    \vspace{0.5ex}

\hspace{.8\funcindent}\begin{boxedminipage}{\funcwidth}

    \raggedright \textbf{leerTweetsAnimales}(\textit{self}, \textit{usu})

    \vspace{-1.5ex}

    \rule{\textwidth}{0.5\fboxrule}
\setlength{\parskip}{2ex}
    Esta función se encarga de leer los tweets de la categoría animales 
    para el usuario \textbf{usu}. Una vez que se accede al archivo, estos 
    tweets, serán almacenados en una lista para después poder mostrarlos. 
    En caso de que el archivo que buscamos no exista, la lista quedará 
    vacia, o lo que es lo mismo, que no hay tweets en esta categoría.

\setlength{\parskip}{1ex}
      \textbf{Parameters}
      \vspace{-1ex}

      \begin{quote}
        \begin{Ventry}{xxx}

          \item[usu]

          Usuario al que queremos leer sus tweets de la categoría animales.

            {\it (type=str)}

        \end{Ventry}

      \end{quote}

      \textbf{Return Value}
    \vspace{-1ex}

      \begin{quote}
      Devuelve una lista \textbf{animales}.

      {\it (type=list)}

      \end{quote}

    \end{boxedminipage}

    \label{funcionesTwitter:FuncionesTwitter:obtenerTweetsComida}
    \index{funcionesTwitter \textit{(module)}!funcionesTwitter.FuncionesTwitter \textit{(class)}!funcionesTwitter.FuncionesTwitter.obtenerTweetsComida \textit{(method)}}

    \vspace{0.5ex}

\hspace{.8\funcindent}\begin{boxedminipage}{\funcwidth}

    \raggedright \textbf{obtenerTweetsComida}(\textit{self}, \textit{usu})

    \vspace{-1.5ex}

    \rule{\textwidth}{0.5\fboxrule}
\setlength{\parskip}{2ex}
    Esta función se encarga de obtener los tweets relacionados con la 
    comida para el usuario \textbf{usu}. Para ello, se llamamos a la 
    función \textit{leerTweetsComida(usu)}, y una vez leidos, creamos la 
    ventana con estos tweets.

\setlength{\parskip}{1ex}
      \textbf{Parameters}
      \vspace{-1ex}

      \begin{quote}
        \begin{Ventry}{xxx}

          \item[usu]

          Usuario del que queremos conocer sus tweets de comida.

            {\it (type=str)}

        \end{Ventry}

      \end{quote}

    \end{boxedminipage}

    \label{funcionesTwitter:FuncionesTwitter:leerTweetsComida}
    \index{funcionesTwitter \textit{(module)}!funcionesTwitter.FuncionesTwitter \textit{(class)}!funcionesTwitter.FuncionesTwitter.leerTweetsComida \textit{(method)}}

    \vspace{0.5ex}

\hspace{.8\funcindent}\begin{boxedminipage}{\funcwidth}

    \raggedright \textbf{leerTweetsComida}(\textit{self}, \textit{usu})

    \vspace{-1.5ex}

    \rule{\textwidth}{0.5\fboxrule}
\setlength{\parskip}{2ex}
    Esta función se encarga de leer los tweets de la categoría comida para 
    el usuario \textbf{usu}. Una vez que se accede al archivo, estos 
    tweets, serán almacenados en una lista para después poder mostrarlos. 
    En caso de que el archivo que buscamos no exista, la lista quedará 
    vacia, o lo que es lo mismo, que no hay tweets en esta categoría.

\setlength{\parskip}{1ex}
      \textbf{Parameters}
      \vspace{-1ex}

      \begin{quote}
        \begin{Ventry}{xxx}

          \item[usu]

          Usuario al que queremos leer sus tweets de la categoría comida.

            {\it (type=str)}

        \end{Ventry}

      \end{quote}

      \textbf{Return Value}
    \vspace{-1ex}

      \begin{quote}
      Devuelve una lista \textbf{comida}.

      {\it (type=list)}

      \end{quote}

    \end{boxedminipage}

    \label{funcionesTwitter:FuncionesTwitter:obtenerTweetsRopa}
    \index{funcionesTwitter \textit{(module)}!funcionesTwitter.FuncionesTwitter \textit{(class)}!funcionesTwitter.FuncionesTwitter.obtenerTweetsRopa \textit{(method)}}

    \vspace{0.5ex}

\hspace{.8\funcindent}\begin{boxedminipage}{\funcwidth}

    \raggedright \textbf{obtenerTweetsRopa}(\textit{self}, \textit{usu})

    \vspace{-1.5ex}

    \rule{\textwidth}{0.5\fboxrule}
\setlength{\parskip}{2ex}
    Esta función se encarga de obtener los tweets relacionados con la ropa 
    para el usuario \textbf{usu}. Para ello, se llamamos a la función 
    \textit{leerTweetsRopa(usu)}, y una vez leidos, creamos la ventana con 
    estos tweets.

\setlength{\parskip}{1ex}
      \textbf{Parameters}
      \vspace{-1ex}

      \begin{quote}
        \begin{Ventry}{xxx}

          \item[usu]

          Usuario del que queremos conocer sus tweets de ropa.

            {\it (type=str)}

        \end{Ventry}

      \end{quote}

    \end{boxedminipage}

    \label{funcionesTwitter:FuncionesTwitter:leerTweetsRopa}
    \index{funcionesTwitter \textit{(module)}!funcionesTwitter.FuncionesTwitter \textit{(class)}!funcionesTwitter.FuncionesTwitter.leerTweetsRopa \textit{(method)}}

    \vspace{0.5ex}

\hspace{.8\funcindent}\begin{boxedminipage}{\funcwidth}

    \raggedright \textbf{leerTweetsRopa}(\textit{self}, \textit{usu})

    \vspace{-1.5ex}

    \rule{\textwidth}{0.5\fboxrule}
\setlength{\parskip}{2ex}
    Esta función se encarga de leer los tweets de la categoría ropa para el
    usuario \textbf{usu}. Una vez que se accede al archivo, estos tweets, 
    serán almacenados en una lista para después poder mostrarlos. En caso 
    de que el archivo que buscamos no exista, la lista quedará vacia, o lo 
    que es lo mismo, que no hay tweets en esta categoría.

\setlength{\parskip}{1ex}
      \textbf{Parameters}
      \vspace{-1ex}

      \begin{quote}
        \begin{Ventry}{xxx}

          \item[usu]

          Usuario al que queremos leer sus tweets de la categoría ropa.

            {\it (type=str)}

        \end{Ventry}

      \end{quote}

      \textbf{Return Value}
    \vspace{-1ex}

      \begin{quote}
      Devuelve una lista \textbf{ropa}.

      {\it (type=list)}

      \end{quote}

    \end{boxedminipage}

    \label{funcionesTwitter:FuncionesTwitter:obtenerTweetsTerrorismo}
    \index{funcionesTwitter \textit{(module)}!funcionesTwitter.FuncionesTwitter \textit{(class)}!funcionesTwitter.FuncionesTwitter.obtenerTweetsTerrorismo \textit{(method)}}

    \vspace{0.5ex}

\hspace{.8\funcindent}\begin{boxedminipage}{\funcwidth}

    \raggedright \textbf{obtenerTweetsTerrorismo}(\textit{self}, \textit{usu})

    \vspace{-1.5ex}

    \rule{\textwidth}{0.5\fboxrule}
\setlength{\parskip}{2ex}
    Esta función se encarga de obtener los tweets relacionados con el 
    terrorismo para el usuario \textbf{usu}. Para ello, se llamamos a la 
    función \textit{leerTweetsTerrorismo(usu)}, y una vez leidos, creamos 
    la ventana con estos tweets.

\setlength{\parskip}{1ex}
      \textbf{Parameters}
      \vspace{-1ex}

      \begin{quote}
        \begin{Ventry}{xxx}

          \item[usu]

          Usuario del que queremos conocer sus tweets de terrorismo.

            {\it (type=str)}

        \end{Ventry}

      \end{quote}

    \end{boxedminipage}

    \label{funcionesTwitter:FuncionesTwitter:leerTweetsTerrorismo}
    \index{funcionesTwitter \textit{(module)}!funcionesTwitter.FuncionesTwitter \textit{(class)}!funcionesTwitter.FuncionesTwitter.leerTweetsTerrorismo \textit{(method)}}

    \vspace{0.5ex}

\hspace{.8\funcindent}\begin{boxedminipage}{\funcwidth}

    \raggedright \textbf{leerTweetsTerrorismo}(\textit{self}, \textit{usu})

    \vspace{-1.5ex}

    \rule{\textwidth}{0.5\fboxrule}
\setlength{\parskip}{2ex}
    Esta función se encarga de leer los tweets de la categoría terrorismo 
    para el usuario \textbf{usu}. Una vez que se accede al archivo, estos 
    tweets, serán almacenados en una lista para después poder mostrarlos. 
    En caso de que el archivo que buscamos no exista, la lista quedará 
    vacia, o lo que es lo mismo, que no hay tweets en esta categoría.

\setlength{\parskip}{1ex}
      \textbf{Parameters}
      \vspace{-1ex}

      \begin{quote}
        \begin{Ventry}{xxx}

          \item[usu]

          Usuario al que queremos leer sus tweets de la categoría 
          terrorismo.

            {\it (type=str)}

        \end{Ventry}

      \end{quote}

      \textbf{Return Value}
    \vspace{-1ex}

      \begin{quote}
      Devuelve una lista \textbf{terrorismo}.

      {\it (type=list)}

      \end{quote}

    \end{boxedminipage}

    \label{funcionesTwitter:FuncionesTwitter:obtenerTweetsSc}
    \index{funcionesTwitter \textit{(module)}!funcionesTwitter.FuncionesTwitter \textit{(class)}!funcionesTwitter.FuncionesTwitter.obtenerTweetsSc \textit{(method)}}

    \vspace{0.5ex}

\hspace{.8\funcindent}\begin{boxedminipage}{\funcwidth}

    \raggedright \textbf{obtenerTweetsSc}(\textit{self}, \textit{usu})

    \vspace{-1.5ex}

    \rule{\textwidth}{0.5\fboxrule}
\setlength{\parskip}{2ex}
    Esta función se encarga de obtener los tweets relacionados sin 
    calificar para el usuario \textbf{usu}. Para ello, se llamamos a la 
    función \textit{leerTweetsSc(usu)}, y una vez leidos, creamos la 
    ventana con estos tweets.

\setlength{\parskip}{1ex}
      \textbf{Parameters}
      \vspace{-1ex}

      \begin{quote}
        \begin{Ventry}{xxx}

          \item[usu]

          Usuario del que queremos conocer sus tweets sin calificar.

            {\it (type=str)}

        \end{Ventry}

      \end{quote}

    \end{boxedminipage}

    \label{funcionesTwitter:FuncionesTwitter:leerTweetsSc}
    \index{funcionesTwitter \textit{(module)}!funcionesTwitter.FuncionesTwitter \textit{(class)}!funcionesTwitter.FuncionesTwitter.leerTweetsSc \textit{(method)}}

    \vspace{0.5ex}

\hspace{.8\funcindent}\begin{boxedminipage}{\funcwidth}

    \raggedright \textbf{leerTweetsSc}(\textit{self}, \textit{usu})

    \vspace{-1.5ex}

    \rule{\textwidth}{0.5\fboxrule}
\setlength{\parskip}{2ex}
    Esta función se encarga de leer los tweets de la categoría sin 
    calificar para el usuario \textbf{usu}. Una vez que se accede al 
    archivo, estos tweets, serán almacenados en una lista para después 
    poder mostrarlos. En caso de que el archivo que buscamos no exista, la 
    lista quedará vacia, o lo que es lo mismo, que no hay tweets en esta 
    categoría.

\setlength{\parskip}{1ex}
      \textbf{Parameters}
      \vspace{-1ex}

      \begin{quote}
        \begin{Ventry}{xxx}

          \item[usu]

          Usuario al que queremos leer sus tweets de la categoría sin 
          calificar.

            {\it (type=str)}

        \end{Ventry}

      \end{quote}

      \textbf{Return Value}
    \vspace{-1ex}

      \begin{quote}
      Devuelve una lista \textbf{sc}.

      {\it (type=list)}

      \end{quote}

    \end{boxedminipage}

    \label{funcionesTwitter:FuncionesTwitter:obtenerDiccionarioAnimales}
    \index{funcionesTwitter \textit{(module)}!funcionesTwitter.FuncionesTwitter \textit{(class)}!funcionesTwitter.FuncionesTwitter.obtenerDiccionarioAnimales \textit{(method)}}

    \vspace{0.5ex}

\hspace{.8\funcindent}\begin{boxedminipage}{\funcwidth}

    \raggedright \textbf{obtenerDiccionarioAnimales}(\textit{self})

    \vspace{-1.5ex}

    \rule{\textwidth}{0.5\fboxrule}
\setlength{\parskip}{2ex}
    Esta función se encarga de leer el diccionario de animales y almacenar 
    sus términos en una lista.

\setlength{\parskip}{1ex}
      \textbf{Return Value}
    \vspace{-1ex}

      \begin{quote}
      Devuelve \textbf{listaAnimales}.

      {\it (type=list)}

      \end{quote}

    \end{boxedminipage}

    \label{funcionesTwitter:FuncionesTwitter:obtenerDiccionarioRopa}
    \index{funcionesTwitter \textit{(module)}!funcionesTwitter.FuncionesTwitter \textit{(class)}!funcionesTwitter.FuncionesTwitter.obtenerDiccionarioRopa \textit{(method)}}

    \vspace{0.5ex}

\hspace{.8\funcindent}\begin{boxedminipage}{\funcwidth}

    \raggedright \textbf{obtenerDiccionarioRopa}(\textit{self})

    \vspace{-1.5ex}

    \rule{\textwidth}{0.5\fboxrule}
\setlength{\parskip}{2ex}
    Esta función se encarga de leer el diccionario de ropa y almacenar sus 
    términos en una lista.

\setlength{\parskip}{1ex}
      \textbf{Return Value}
    \vspace{-1ex}

      \begin{quote}
      Devuelve \textbf{listaRopa}.

      {\it (type=list)}

      \end{quote}

    \end{boxedminipage}

    \label{funcionesTwitter:FuncionesTwitter:obtenerDiccionarioComida}
    \index{funcionesTwitter \textit{(module)}!funcionesTwitter.FuncionesTwitter \textit{(class)}!funcionesTwitter.FuncionesTwitter.obtenerDiccionarioComida \textit{(method)}}

    \vspace{0.5ex}

\hspace{.8\funcindent}\begin{boxedminipage}{\funcwidth}

    \raggedright \textbf{obtenerDiccionarioComida}(\textit{self})

    \vspace{-1.5ex}

    \rule{\textwidth}{0.5\fboxrule}
\setlength{\parskip}{2ex}
    Esta función se encarga de leer el diccionario de comida y almacenar 
    sus términos en una lista.

\setlength{\parskip}{1ex}
      \textbf{Return Value}
    \vspace{-1ex}

      \begin{quote}
      Devuelve \textbf{listaComida}.

      {\it (type=list)}

      \end{quote}

    \end{boxedminipage}

    \label{funcionesTwitter:FuncionesTwitter:obtenerDiccionarioTerrorismo}
    \index{funcionesTwitter \textit{(module)}!funcionesTwitter.FuncionesTwitter \textit{(class)}!funcionesTwitter.FuncionesTwitter.obtenerDiccionarioTerrorismo \textit{(method)}}

    \vspace{0.5ex}

\hspace{.8\funcindent}\begin{boxedminipage}{\funcwidth}

    \raggedright \textbf{obtenerDiccionarioTerrorismo}(\textit{self})

    \vspace{-1.5ex}

    \rule{\textwidth}{0.5\fboxrule}
\setlength{\parskip}{2ex}
    Esta función se encarga de leer el diccionario de terrorismo y 
    almacenar sus términos en una lista.

\setlength{\parskip}{1ex}
      \textbf{Return Value}
    \vspace{-1ex}

      \begin{quote}
      Devuelve \textbf{listaTerrorismo}.

      {\it (type=list)}

      \end{quote}

    \end{boxedminipage}

    \label{funcionesTwitter:FuncionesTwitter:obtenerTweetsUsuario}
    \index{funcionesTwitter \textit{(module)}!funcionesTwitter.FuncionesTwitter \textit{(class)}!funcionesTwitter.FuncionesTwitter.obtenerTweetsUsuario \textit{(method)}}

    \vspace{0.5ex}

\hspace{.8\funcindent}\begin{boxedminipage}{\funcwidth}

    \raggedright \textbf{obtenerTweetsUsuario}(\textit{self}, \textit{usuario})

    \vspace{-1.5ex}

    \rule{\textwidth}{0.5\fboxrule}
\setlength{\parskip}{2ex}
    Esta función se encarga de leer todos los tweets de un usuario siempre 
    que sea permitido por la API.

\setlength{\parskip}{1ex}
      \textbf{Parameters}
      \vspace{-1ex}

      \begin{quote}
        \begin{Ventry}{xxxxxxx}

          \item[usuario]

          Usuario del que queremos obtener sus tweets.

            {\it (type=str)}

        \end{Ventry}

      \end{quote}

      \textbf{Return Value}
    \vspace{-1ex}

      \begin{quote}
      Devuelve \textbf{salida}, con los tweets de un usuario dado.

      {\it (type=list)}

      \end{quote}

    \end{boxedminipage}

    \index{funcionesTwitter \textit{(module)}!funcionesTwitter.FuncionesTwitter \textit{(class)}|)}
    \index{funcionesTwitter \textit{(module)}|)}
