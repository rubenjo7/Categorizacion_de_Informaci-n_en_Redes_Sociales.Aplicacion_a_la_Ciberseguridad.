%
% API Documentation for Categorizacion de Informacion en Redes Sociales. Aplicacion a la Ciberseguridad.
% Module categoria
%
% Generated by epydoc 3.0.1
% [Sat Jun 10 22:41:26 2017]
%

%%%%%%%%%%%%%%%%%%%%%%%%%%%%%%%%%%%%%%%%%%%%%%%%%%%%%%%%%%%%%%%%%%%%%%%%%%%
%%                          Module Description                           %%
%%%%%%%%%%%%%%%%%%%%%%%%%%%%%%%%%%%%%%%%%%%%%%%%%%%%%%%%%%%%%%%%%%%%%%%%%%%

    \index{categoria \textit{(module)}|(}
\section{Module categoria}

    \label{categoria}
Este módulo contiene la clase \textbf{Categoria}.

\textbf{Author:} Rubén Jiménez Ortega.



\textbf{Version:} 1.0



\textbf{Copyright:} Copyright (C) 2017 by Rubén Jiménez Ortega.




%%%%%%%%%%%%%%%%%%%%%%%%%%%%%%%%%%%%%%%%%%%%%%%%%%%%%%%%%%%%%%%%%%%%%%%%%%%
%%                           Class Description                           %%
%%%%%%%%%%%%%%%%%%%%%%%%%%%%%%%%%%%%%%%%%%%%%%%%%%%%%%%%%%%%%%%%%%%%%%%%%%%

    \index{categoria \textit{(module)}!categoria.Categoria \textit{(class)}|(}
\subsection{Class Categoria}

    \label{categoria:Categoria}
\begin{tabular}{cccccc}
% Line for ??-2, linespec=[False]
\multicolumn{2}{r}{\settowidth{\BCL}{??-2}\multirow{2}{\BCL}{??-2}}
&&
  \\\cline{3-3}
  &&\multicolumn{1}{c|}{}
&&
  \\
&&\multicolumn{2}{l}{\textbf{categoria.Categoria}}
\end{tabular}

Crea un QWidget para mostrar los tweets relacionados con una categoría en 
concreto.


%%%%%%%%%%%%%%%%%%%%%%%%%%%%%%%%%%%%%%%%%%%%%%%%%%%%%%%%%%%%%%%%%%%%%%%%%%%
%%                                Methods                                %%
%%%%%%%%%%%%%%%%%%%%%%%%%%%%%%%%%%%%%%%%%%%%%%%%%%%%%%%%%%%%%%%%%%%%%%%%%%%

  \subsubsection{Methods}

    \label{categoria:Categoria:__init__}
    \index{categoria \textit{(module)}!categoria.Categoria \textit{(class)}!categoria.Categoria.\_\_init\_\_ \textit{(method)}}

    \vspace{0.5ex}

\hspace{.8\funcindent}\begin{boxedminipage}{\funcwidth}

    \raggedright \textbf{\_\_init\_\_}(\textit{self}, \textit{titulo}, \textit{categoria}, \textit{contador}, \textit{tweets})

    \vspace{-1.5ex}

    \rule{\textwidth}{0.5\fboxrule}
\setlength{\parskip}{2ex}
    Constructor con parámetros de la clase \textbf{Categoria}.

\setlength{\parskip}{1ex}
      \textbf{Parameters}
      \vspace{-1ex}

      \begin{quote}
        \begin{Ventry}{xxxxxxxxx}

          \item[titulo]

          Variable para cambiar el título de la nueva ventana.

            {\it (type=str)}

          \item[categoria]

          Variable que almacena la categoría que mostramos.

            {\it (type=str)}

          \item[contador]

          Número de tweets de la categoría que estudiamos.

            {\it (type=int)}

          \item[tweets]

          Tweets del usuario y de la categoría elegida almacenados en una 
          lista para poder mostrarlos.

            {\it (type=list)}

        \end{Ventry}

      \end{quote}

    \end{boxedminipage}

    \label{categoria:Categoria:center}
    \index{categoria \textit{(module)}!categoria.Categoria \textit{(class)}!categoria.Categoria.center \textit{(method)}}

    \vspace{0.5ex}

\hspace{.8\funcindent}\begin{boxedminipage}{\funcwidth}

    \raggedright \textbf{center}(\textit{self})

    \vspace{-1.5ex}

    \rule{\textwidth}{0.5\fboxrule}
\setlength{\parskip}{2ex}
    Función para centrar la ventana categoría en una pantalla, 
    independientemente del tamaño de la misma.

\setlength{\parskip}{1ex}
    \end{boxedminipage}

    \index{categoria \textit{(module)}!categoria.Categoria \textit{(class)}|)}
    \index{categoria \textit{(module)}|)}
